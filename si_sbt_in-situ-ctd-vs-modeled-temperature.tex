% Options for packages loaded elsewhere
\PassOptionsToPackage{unicode}{hyperref}
\PassOptionsToPackage{hyphens}{url}
%
\documentclass[
]{article}
\usepackage{amsmath,amssymb}
\usepackage{lmodern}
\usepackage{iftex}
\ifPDFTeX
  \usepackage[T1]{fontenc}
  \usepackage[utf8]{inputenc}
  \usepackage{textcomp} % provide euro and other symbols
\else % if luatex or xetex
  \usepackage{unicode-math}
  \defaultfontfeatures{Scale=MatchLowercase}
  \defaultfontfeatures[\rmfamily]{Ligatures=TeX,Scale=1}
\fi
% Use upquote if available, for straight quotes in verbatim environments
\IfFileExists{upquote.sty}{\usepackage{upquote}}{}
\IfFileExists{microtype.sty}{% use microtype if available
  \usepackage[]{microtype}
  \UseMicrotypeSet[protrusion]{basicmath} % disable protrusion for tt fonts
}{}
\makeatletter
\@ifundefined{KOMAClassName}{% if non-KOMA class
  \IfFileExists{parskip.sty}{%
    \usepackage{parskip}
  }{% else
    \setlength{\parindent}{0pt}
    \setlength{\parskip}{6pt plus 2pt minus 1pt}}
}{% if KOMA class
  \KOMAoptions{parskip=half}}
\makeatother
\usepackage{xcolor}
\usepackage[margin=1in]{geometry}
\usepackage{color}
\usepackage{fancyvrb}
\newcommand{\VerbBar}{|}
\newcommand{\VERB}{\Verb[commandchars=\\\{\}]}
\DefineVerbatimEnvironment{Highlighting}{Verbatim}{commandchars=\\\{\}}
% Add ',fontsize=\small' for more characters per line
\usepackage{framed}
\definecolor{shadecolor}{RGB}{248,248,248}
\newenvironment{Shaded}{\begin{snugshade}}{\end{snugshade}}
\newcommand{\AlertTok}[1]{\textcolor[rgb]{0.94,0.16,0.16}{#1}}
\newcommand{\AnnotationTok}[1]{\textcolor[rgb]{0.56,0.35,0.01}{\textbf{\textit{#1}}}}
\newcommand{\AttributeTok}[1]{\textcolor[rgb]{0.77,0.63,0.00}{#1}}
\newcommand{\BaseNTok}[1]{\textcolor[rgb]{0.00,0.00,0.81}{#1}}
\newcommand{\BuiltInTok}[1]{#1}
\newcommand{\CharTok}[1]{\textcolor[rgb]{0.31,0.60,0.02}{#1}}
\newcommand{\CommentTok}[1]{\textcolor[rgb]{0.56,0.35,0.01}{\textit{#1}}}
\newcommand{\CommentVarTok}[1]{\textcolor[rgb]{0.56,0.35,0.01}{\textbf{\textit{#1}}}}
\newcommand{\ConstantTok}[1]{\textcolor[rgb]{0.00,0.00,0.00}{#1}}
\newcommand{\ControlFlowTok}[1]{\textcolor[rgb]{0.13,0.29,0.53}{\textbf{#1}}}
\newcommand{\DataTypeTok}[1]{\textcolor[rgb]{0.13,0.29,0.53}{#1}}
\newcommand{\DecValTok}[1]{\textcolor[rgb]{0.00,0.00,0.81}{#1}}
\newcommand{\DocumentationTok}[1]{\textcolor[rgb]{0.56,0.35,0.01}{\textbf{\textit{#1}}}}
\newcommand{\ErrorTok}[1]{\textcolor[rgb]{0.64,0.00,0.00}{\textbf{#1}}}
\newcommand{\ExtensionTok}[1]{#1}
\newcommand{\FloatTok}[1]{\textcolor[rgb]{0.00,0.00,0.81}{#1}}
\newcommand{\FunctionTok}[1]{\textcolor[rgb]{0.00,0.00,0.00}{#1}}
\newcommand{\ImportTok}[1]{#1}
\newcommand{\InformationTok}[1]{\textcolor[rgb]{0.56,0.35,0.01}{\textbf{\textit{#1}}}}
\newcommand{\KeywordTok}[1]{\textcolor[rgb]{0.13,0.29,0.53}{\textbf{#1}}}
\newcommand{\NormalTok}[1]{#1}
\newcommand{\OperatorTok}[1]{\textcolor[rgb]{0.81,0.36,0.00}{\textbf{#1}}}
\newcommand{\OtherTok}[1]{\textcolor[rgb]{0.56,0.35,0.01}{#1}}
\newcommand{\PreprocessorTok}[1]{\textcolor[rgb]{0.56,0.35,0.01}{\textit{#1}}}
\newcommand{\RegionMarkerTok}[1]{#1}
\newcommand{\SpecialCharTok}[1]{\textcolor[rgb]{0.00,0.00,0.00}{#1}}
\newcommand{\SpecialStringTok}[1]{\textcolor[rgb]{0.31,0.60,0.02}{#1}}
\newcommand{\StringTok}[1]{\textcolor[rgb]{0.31,0.60,0.02}{#1}}
\newcommand{\VariableTok}[1]{\textcolor[rgb]{0.00,0.00,0.00}{#1}}
\newcommand{\VerbatimStringTok}[1]{\textcolor[rgb]{0.31,0.60,0.02}{#1}}
\newcommand{\WarningTok}[1]{\textcolor[rgb]{0.56,0.35,0.01}{\textbf{\textit{#1}}}}
\usepackage{graphicx}
\makeatletter
\def\maxwidth{\ifdim\Gin@nat@width>\linewidth\linewidth\else\Gin@nat@width\fi}
\def\maxheight{\ifdim\Gin@nat@height>\textheight\textheight\else\Gin@nat@height\fi}
\makeatother
% Scale images if necessary, so that they will not overflow the page
% margins by default, and it is still possible to overwrite the defaults
% using explicit options in \includegraphics[width, height, ...]{}
\setkeys{Gin}{width=\maxwidth,height=\maxheight,keepaspectratio}
% Set default figure placement to htbp
\makeatletter
\def\fps@figure{htbp}
\makeatother
\setlength{\emergencystretch}{3em} % prevent overfull lines
\providecommand{\tightlist}{%
  \setlength{\itemsep}{0pt}\setlength{\parskip}{0pt}}
\setcounter{secnumdepth}{-\maxdimen} % remove section numbering
\ifLuaTeX
  \usepackage{selnolig}  % disable illegal ligatures
\fi
\IfFileExists{bookmark.sty}{\usepackage{bookmark}}{\usepackage{hyperref}}
\IfFileExists{xurl.sty}{\usepackage{xurl}}{} % add URL line breaks if available
\urlstyle{same} % disable monospaced font for URLs
\hypersetup{
  pdftitle={ices ctd bottom temperature process},
  hidelinks,
  pdfcreator={LaTeX via pandoc}}

\title{ices ctd bottom temperature process}
\author{}
\date{\vspace{-2.5em}}

\begin{document}
\maketitle

\hypertarget{info}{%
\subsection{INFO}\label{info}}

TEAMPERATURE DATA EXPLORATION (ICES CTD, ISIMIP SBT, HADISST)

PROJECT: FWO PHD - WP1

Tuan Anh Bui (23/02/2023)

Compare ICES CTD data and ISIMIP Sea bottom temperature vs HADISST sea
surface temperature

\hypertarget{setup}{%
\subsubsection{1. SETUP}\label{setup}}

\begin{Shaded}
\begin{Highlighting}[]
\FunctionTok{library}\NormalTok{(tidyverse)  }\CommentTok{\# process dataframe}
\end{Highlighting}
\end{Shaded}

\begin{verbatim}
## Warning: package 'tidyverse' was built under R version 4.2.2
\end{verbatim}

\begin{verbatim}
## -- Attaching packages --------------------------------------- tidyverse 1.3.2 --
## v ggplot2 3.4.1     v purrr   0.3.4
## v tibble  3.1.7     v dplyr   1.0.9
## v tidyr   1.2.0     v stringr 1.4.0
## v readr   2.1.2     v forcats 0.5.1
\end{verbatim}

\begin{verbatim}
## Warning: package 'ggplot2' was built under R version 4.2.2
\end{verbatim}

\begin{verbatim}
## -- Conflicts ------------------------------------------ tidyverse_conflicts() --
## x dplyr::filter() masks stats::filter()
## x dplyr::lag()    masks stats::lag()
\end{verbatim}

\begin{Shaded}
\begin{Highlighting}[]
\FunctionTok{library}\NormalTok{(lubridate)  }\CommentTok{\# process date time }
\end{Highlighting}
\end{Shaded}

\begin{verbatim}
## 
## Attaching package: 'lubridate'
## 
## The following objects are masked from 'package:base':
## 
##     date, intersect, setdiff, union
\end{verbatim}

\begin{Shaded}
\begin{Highlighting}[]
\FunctionTok{library}\NormalTok{(readr)      }\CommentTok{\# read csv}
\FunctionTok{library}\NormalTok{(sf)         }\CommentTok{\# process GIS files}
\end{Highlighting}
\end{Shaded}

\begin{verbatim}
## Warning: package 'sf' was built under R version 4.2.2
\end{verbatim}

\begin{verbatim}
## Linking to GEOS 3.9.3, GDAL 3.5.2, PROJ 8.2.1; sf_use_s2() is TRUE
\end{verbatim}

\begin{Shaded}
\begin{Highlighting}[]
\FunctionTok{library}\NormalTok{(stars)      }\CommentTok{\# process raster files}
\end{Highlighting}
\end{Shaded}

\begin{verbatim}
## Warning: package 'stars' was built under R version 4.2.2
\end{verbatim}

\begin{verbatim}
## Loading required package: abind
\end{verbatim}

\begin{Shaded}
\begin{Highlighting}[]
\FunctionTok{library}\NormalTok{(viridis)    }\CommentTok{\# color scale}
\end{Highlighting}
\end{Shaded}

\begin{verbatim}
## Loading required package: viridisLite
\end{verbatim}

\begin{verbatim}
## Warning: package 'viridisLite' was built under R version 4.2.3
\end{verbatim}

\hypertarget{read-data}{%
\subsubsection{2. READ DATA}\label{read-data}}

\begin{Shaded}
\begin{Highlighting}[]
\CommentTok{\# ICES AREA}
\NormalTok{dir\_gis }\OtherTok{\textless{}{-}} \StringTok{"D:/OneDrive {-} UGent/data/Admin"}
\NormalTok{file\_gis }\OtherTok{\textless{}{-}} \StringTok{"ices\_areas\_sub\_group\_4326\_new.gpkg"}

\NormalTok{ices\_area }\OtherTok{\textless{}{-}} \FunctionTok{read\_sf}\NormalTok{(}\FunctionTok{file.path}\NormalTok{(dir\_gis, file\_gis))}

\NormalTok{ices\_area }\OtherTok{\textless{}{-}}\NormalTok{ ices\_area }\SpecialCharTok{\%\textgreater{}\%} \FunctionTok{select}\NormalTok{(Area\_27) }\SpecialCharTok{\%\textgreater{}\%} \FunctionTok{filter}\NormalTok{(Area\_27 }\SpecialCharTok{\%in\%} \FunctionTok{c}\NormalTok{(}\StringTok{"7a"}\NormalTok{, }\StringTok{"8ab"}\NormalTok{, }\StringTok{"4bc"}\NormalTok{))}

\CommentTok{\# CTD}
\NormalTok{dir\_ctd }\OtherTok{\textless{}{-}} \StringTok{"D:/OneDrive {-} UGent/data/Env/Sea Bottom Temperature\_ICES CTD\_1970{-}2020"}
\NormalTok{file\_ctd }\OtherTok{\textless{}{-}} \StringTok{"ices\_ctd\_sea{-}bottom{-}temperature.gpkg"}

\NormalTok{ctd }\OtherTok{\textless{}{-}} \FunctionTok{read\_sf}\NormalTok{(}\FunctionTok{file.path}\NormalTok{(dir\_ctd, file\_ctd))}
\NormalTok{ctd }\OtherTok{\textless{}{-}}\NormalTok{ ctd }\SpecialCharTok{\%\textgreater{}\%} \FunctionTok{filter}\NormalTok{(Area\_27 }\SpecialCharTok{\%in\%} \FunctionTok{c}\NormalTok{(}\StringTok{"7a"}\NormalTok{, }\StringTok{"8ab"}\NormalTok{, }\StringTok{"4bc"}\NormalTok{))}
\end{Highlighting}
\end{Shaded}

\hypertarget{compare-ices-ctd-to-isimip-thetao-isimip-tob-and-hadisst}{%
\subsubsection{3. COMPARE ICES CTD TO ISIMIP THETAO, ISIMIP TOB, AND
HADISST}\label{compare-ices-ctd-to-isimip-thetao-isimip-tob-and-hadisst}}

\hypertarget{isimip-thetao}{%
\paragraph{3.1. ISIMIP THETAO}\label{isimip-thetao}}

ISIMIP thetao - potential sea water temperature. The original data has
temperature at depth but was processed to obtain sea bottom temperature

\hypertarget{process-data}{%
\subparagraph{Process data}\label{process-data}}

\begin{Shaded}
\begin{Highlighting}[]
\CommentTok{\# load data}

\NormalTok{dir\_thetao }\OtherTok{\textless{}{-}} \StringTok{"D:/OneDrive {-} UGent/data/Env/Sea Temperature\_ISIMIP3\_0.5deg\_1850{-}2014"}
\NormalTok{file\_thetao }\OtherTok{\textless{}{-}} \StringTok{"sea bottom temperature\_mpi{-}esm\_05deg.tif"}

\NormalTok{thetao }\OtherTok{\textless{}{-}} \FunctionTok{read\_stars}\NormalTok{(}\FunctionTok{file.path}\NormalTok{(dir\_thetao, file\_thetao))}

\CommentTok{\# Extract values of thetao at ctd points}
\CommentTok{\# change time values thetao}
\NormalTok{year\_thetao }\OtherTok{\textless{}{-}} \FunctionTok{str\_sub}\NormalTok{(}\FunctionTok{attr}\NormalTok{(thetao, }\StringTok{"dimensions"}\NormalTok{)}\SpecialCharTok{$}\NormalTok{band}\SpecialCharTok{$}\NormalTok{values, }\AttributeTok{start =} \DecValTok{4}\NormalTok{, }\AttributeTok{end =} \DecValTok{7}\NormalTok{)}
\NormalTok{month\_thetao }\OtherTok{\textless{}{-}} \FunctionTok{str\_sub}\NormalTok{(}\FunctionTok{attr}\NormalTok{(thetao, }\StringTok{"dimensions"}\NormalTok{)}\SpecialCharTok{$}\NormalTok{band}\SpecialCharTok{$}\NormalTok{values, }\AttributeTok{start =} \DecValTok{9}\NormalTok{, }\AttributeTok{end =} \DecValTok{10}\NormalTok{)}
\NormalTok{date\_thetao }\OtherTok{\textless{}{-}} \FunctionTok{as.Date}\NormalTok{(}\FunctionTok{paste0}\NormalTok{(}\StringTok{"01"}\NormalTok{, }\StringTok{"{-}"}\NormalTok{, month\_thetao, }\StringTok{"{-}"}\NormalTok{, year\_thetao),}\AttributeTok{format =} \StringTok{"\%d{-}\%m{-}\%Y"}\NormalTok{)}

\CommentTok{\# change band dimension to time dimension}
\NormalTok{thetao }\OtherTok{\textless{}{-}} \FunctionTok{st\_set\_dimensions}\NormalTok{(thetao, }\DecValTok{3}\NormalTok{, }\AttributeTok{values =}\NormalTok{ date\_thetao, }\AttributeTok{names =} \StringTok{"time"}\NormalTok{)}

\CommentTok{\# change time ctd\_test}
\NormalTok{ctd }\OtherTok{\textless{}{-}}\NormalTok{ ctd }\SpecialCharTok{\%\textgreater{}\%} \FunctionTok{mutate}\NormalTok{(}\AttributeTok{date\_match =} \FunctionTok{as.Date}\NormalTok{(}\FunctionTok{paste0}\NormalTok{(}\StringTok{"01"}\NormalTok{, }\StringTok{"{-}"}\NormalTok{, Month, }\StringTok{"{-}"}\NormalTok{, Year),}\AttributeTok{format =} \StringTok{"\%d{-}\%m{-}\%Y"}\NormalTok{))}

\CommentTok{\# extract }
\NormalTok{thetao\_extract }\OtherTok{\textless{}{-}} \FunctionTok{st\_extract}\NormalTok{(thetao, ctd, }\AttributeTok{time\_column =} \StringTok{"date\_match"}\NormalTok{)}

\CommentTok{\# joint with ctd data for comparison}
\NormalTok{thetao\_join }\OtherTok{\textless{}{-}} \FunctionTok{left\_join}\NormalTok{(}\FunctionTok{as.data.frame}\NormalTok{(ctd), }\FunctionTok{as.data.frame}\NormalTok{(thetao\_extract))}
\end{Highlighting}
\end{Shaded}

\begin{verbatim}
## Joining, by = c("geom", "date_match")
\end{verbatim}

\begin{Shaded}
\begin{Highlighting}[]
\NormalTok{thetao\_join}\SpecialCharTok{$}\NormalTok{thetao }\OtherTok{\textless{}{-}}\NormalTok{ thetao\_join}\SpecialCharTok{$}\StringTok{\textasciigrave{}}\AttributeTok{sea bottom temperature\_mpi{-}esm\_05deg.tif}\StringTok{\textasciigrave{}}

\CommentTok{\# the rows of sbt\_join differs from ctd and sbt\_extract due to the lack of Cruise and Station information in joining}

\NormalTok{thetao\_join }\OtherTok{\textless{}{-}}\NormalTok{ thetao\_join }\SpecialCharTok{\%\textgreater{}\%} \FunctionTok{group\_by}\NormalTok{(ID, Cruise, Station, Year, Month, Day, sample\_date, Area\_27, date\_match, geom, depth\_m) }\SpecialCharTok{\%\textgreater{}\%} \FunctionTok{summarize}\NormalTok{(}\AttributeTok{thetao =} \FunctionTok{mean}\NormalTok{(thetao), }\AttributeTok{temp\_degC =} \FunctionTok{mean}\NormalTok{(temp\_degC))}
\end{Highlighting}
\end{Shaded}

\begin{verbatim}
## `summarise()` has grouped output by 'ID', 'Cruise', 'Station', 'Year', 'Month',
## 'Day', 'sample_date', 'Area_27', 'date_match', 'geom'. You can override using
## the `.groups` argument.
\end{verbatim}

\hypertarget{correlation-test}{%
\subparagraph{Correlation test}\label{correlation-test}}

At CTD locations

\begin{Shaded}
\begin{Highlighting}[]
\CommentTok{\# Correlation test }

\CommentTok{\# all data}
\FunctionTok{cor.test}\NormalTok{(thetao\_join}\SpecialCharTok{$}\NormalTok{temp\_degC, thetao\_join}\SpecialCharTok{$}\NormalTok{thetao) }\CommentTok{\#0.900 [95CI 0.900{-}0.902]}
\end{Highlighting}
\end{Shaded}

\begin{verbatim}
## 
##  Pearson's product-moment correlation
## 
## data:  thetao_join$temp_degC and thetao_join$thetao
## t = 393.18, df = 36236, p-value < 2.2e-16
## alternative hypothesis: true correlation is not equal to 0
## 95 percent confidence interval:
##  0.8980872 0.9019978
## sample estimates:
##       cor 
## 0.9000606
\end{verbatim}

\begin{Shaded}
\begin{Highlighting}[]
\CommentTok{\# by ices division}
\NormalTok{thetao\_4bc }\OtherTok{\textless{}{-}}\NormalTok{ thetao\_join }\SpecialCharTok{\%\textgreater{}\%} \FunctionTok{filter}\NormalTok{(Area\_27 }\SpecialCharTok{==} \StringTok{"4bc"}\NormalTok{)}
\NormalTok{thetao\_7a }\OtherTok{\textless{}{-}}\NormalTok{ thetao\_join }\SpecialCharTok{\%\textgreater{}\%} \FunctionTok{filter}\NormalTok{(Area\_27 }\SpecialCharTok{==} \StringTok{"7a"}\NormalTok{)}
\NormalTok{thetao\_8ab }\OtherTok{\textless{}{-}}\NormalTok{ thetao\_join }\SpecialCharTok{\%\textgreater{}\%} \FunctionTok{filter}\NormalTok{(Area\_27 }\SpecialCharTok{==} \StringTok{"8ab"}\NormalTok{)}

\FunctionTok{cor.test}\NormalTok{(thetao\_4bc}\SpecialCharTok{$}\NormalTok{temp\_degC, thetao\_4bc}\SpecialCharTok{$}\NormalTok{thetao) }\CommentTok{\#0.903 [95CI 0.901{-}0.905]}
\end{Highlighting}
\end{Shaded}

\begin{verbatim}
## 
##  Pearson's product-moment correlation
## 
## data:  thetao_4bc$temp_degC and thetao_4bc$thetao
## t = 380.94, df = 32746, p-value < 2.2e-16
## alternative hypothesis: true correlation is not equal to 0
## 95 percent confidence interval:
##  0.9012539 0.9052423
## sample estimates:
##       cor 
## 0.9032676
\end{verbatim}

\begin{Shaded}
\begin{Highlighting}[]
\FunctionTok{cor.test}\NormalTok{(thetao\_7a}\SpecialCharTok{$}\NormalTok{temp\_degC, thetao\_7a}\SpecialCharTok{$}\NormalTok{thetao)   }\CommentTok{\#0.86 [95CI 0.83{-}0.88]}
\end{Highlighting}
\end{Shaded}

\begin{verbatim}
## 
##  Pearson's product-moment correlation
## 
## data:  thetao_7a$temp_degC and thetao_7a$thetao
## t = 37.691, df = 508, p-value < 2.2e-16
## alternative hypothesis: true correlation is not equal to 0
## 95 percent confidence interval:
##  0.8335380 0.8795345
## sample estimates:
##       cor 
## 0.8582501
\end{verbatim}

\begin{Shaded}
\begin{Highlighting}[]
\FunctionTok{cor.test}\NormalTok{(thetao\_8ab}\SpecialCharTok{$}\NormalTok{temp\_degC, thetao\_8ab}\SpecialCharTok{$}\NormalTok{thetao) }\CommentTok{\#0.38 [95CI 0.35{-}0.41]}
\end{Highlighting}
\end{Shaded}

\begin{verbatim}
## 
##  Pearson's product-moment correlation
## 
## data:  thetao_8ab$temp_degC and thetao_8ab$thetao
## t = 22.467, df = 2978, p-value < 2.2e-16
## alternative hypothesis: true correlation is not equal to 0
## 95 percent confidence interval:
##  0.3495671 0.4109837
## sample estimates:
##       cor 
## 0.3806952
\end{verbatim}

\begin{Shaded}
\begin{Highlighting}[]
\CommentTok{\# plot}
\FunctionTok{ggplot}\NormalTok{(}\AttributeTok{data =}\NormalTok{ thetao\_join, }\FunctionTok{aes}\NormalTok{(}\AttributeTok{x =}\NormalTok{ temp\_degC, }\AttributeTok{y =}\NormalTok{ thetao)) }\SpecialCharTok{+} \FunctionTok{geom\_point}\NormalTok{() }\SpecialCharTok{+} \FunctionTok{geom\_smooth}\NormalTok{(}\AttributeTok{method =} \StringTok{"lm"}\NormalTok{)}
\end{Highlighting}
\end{Shaded}

\begin{verbatim}
## `geom_smooth()` using formula = 'y ~ x'
\end{verbatim}

\begin{verbatim}
## Warning: Removed 11863 rows containing non-finite values (`stat_smooth()`).
\end{verbatim}

\begin{verbatim}
## Warning: Removed 11863 rows containing missing values (`geom_point()`).
\end{verbatim}

\includegraphics{si_sbt_in-situ-ctd-vs-modeled-temperature_files/figure-latex/unnamed-chunk-4-1.pdf}

\begin{Shaded}
\begin{Highlighting}[]
\FunctionTok{ggplot}\NormalTok{(}\AttributeTok{data =}\NormalTok{ thetao\_4bc, }\FunctionTok{aes}\NormalTok{(}\AttributeTok{x =}\NormalTok{ temp\_degC, }\AttributeTok{y =}\NormalTok{ thetao)) }\SpecialCharTok{+} \FunctionTok{geom\_point}\NormalTok{() }\SpecialCharTok{+} \FunctionTok{geom\_smooth}\NormalTok{(}\AttributeTok{method =} \StringTok{"lm"}\NormalTok{) }
\end{Highlighting}
\end{Shaded}

\begin{verbatim}
## `geom_smooth()` using formula = 'y ~ x'
\end{verbatim}

\begin{verbatim}
## Warning: Removed 8300 rows containing non-finite values (`stat_smooth()`).
\end{verbatim}

\begin{verbatim}
## Warning: Removed 8300 rows containing missing values (`geom_point()`).
\end{verbatim}

\includegraphics{si_sbt_in-situ-ctd-vs-modeled-temperature_files/figure-latex/unnamed-chunk-4-2.pdf}

\begin{Shaded}
\begin{Highlighting}[]
\FunctionTok{ggplot}\NormalTok{(}\AttributeTok{data =}\NormalTok{ thetao\_7a, }\FunctionTok{aes}\NormalTok{(}\AttributeTok{x =}\NormalTok{ temp\_degC, }\AttributeTok{y =}\NormalTok{ thetao)) }\SpecialCharTok{+} \FunctionTok{geom\_point}\NormalTok{() }\SpecialCharTok{+} \FunctionTok{geom\_smooth}\NormalTok{(}\AttributeTok{method =} \StringTok{"lm"}\NormalTok{)}
\end{Highlighting}
\end{Shaded}

\begin{verbatim}
## `geom_smooth()` using formula = 'y ~ x'
\end{verbatim}

\begin{verbatim}
## Warning: Removed 2102 rows containing non-finite values (`stat_smooth()`).
\end{verbatim}

\begin{verbatim}
## Warning: Removed 2102 rows containing missing values (`geom_point()`).
\end{verbatim}

\includegraphics{si_sbt_in-situ-ctd-vs-modeled-temperature_files/figure-latex/unnamed-chunk-4-3.pdf}

\begin{Shaded}
\begin{Highlighting}[]
\FunctionTok{ggplot}\NormalTok{(}\AttributeTok{data =}\NormalTok{ thetao\_8ab, }\FunctionTok{aes}\NormalTok{(}\AttributeTok{x =}\NormalTok{ temp\_degC, }\AttributeTok{y =}\NormalTok{ thetao)) }\SpecialCharTok{+} \FunctionTok{geom\_point}\NormalTok{() }\SpecialCharTok{+} \FunctionTok{geom\_smooth}\NormalTok{(}\AttributeTok{method =} \StringTok{"lm"}\NormalTok{)}
\end{Highlighting}
\end{Shaded}

\begin{verbatim}
## `geom_smooth()` using formula = 'y ~ x'
\end{verbatim}

\begin{verbatim}
## Warning: Removed 1461 rows containing non-finite values (`stat_smooth()`).
\end{verbatim}

\begin{verbatim}
## Warning: Removed 1461 rows containing missing values (`geom_point()`).
\end{verbatim}

\includegraphics{si_sbt_in-situ-ctd-vs-modeled-temperature_files/figure-latex/unnamed-chunk-4-4.pdf}

8ab has poor correlation of CTD and ISIMIP thetao (0.38). One potential
issue could be the record of CTD at high depth (\textgreater{} 200m)
while the ISIMIP thetao was constructed at depth \textless{} 200. Thus,
comparison should be confined to records under 200m only. In addition,
sole main distribution is within depth under 200m.

Reference sole distribution:

\# Fishbase - 150m

\# Grati et al 2013 (Adriatic Sea) - 100m

\# Koutsikopoulos and Lacroix 1992 - survey spawning area (Bay of
Biscay) - 100m

\# Maxwell et al 2009 - survey data (Irish Sea) - 130m

\# Pennino et al 2022 - survey data (Iberian water) - 200m

\begin{Shaded}
\begin{Highlighting}[]
\NormalTok{thetao\_8ab\_st200 }\OtherTok{\textless{}{-}}\NormalTok{ thetao\_8ab }\SpecialCharTok{\%\textgreater{}\%} \FunctionTok{filter}\NormalTok{(depth\_m }\SpecialCharTok{\textless{}} \DecValTok{200}\NormalTok{)}

\FunctionTok{cor.test}\NormalTok{(thetao\_8ab\_st200}\SpecialCharTok{$}\NormalTok{temp\_degC, thetao\_8ab\_st200}\SpecialCharTok{$}\NormalTok{thetao) }\CommentTok{\#0.50 [95CI 0.46{-}0.53]}
\end{Highlighting}
\end{Shaded}

\begin{verbatim}
## 
##  Pearson's product-moment correlation
## 
## data:  thetao_8ab_st200$temp_degC and thetao_8ab_st200$thetao
## t = 23.504, df = 1696, p-value < 2.2e-16
## alternative hypothesis: true correlation is not equal to 0
## 95 percent confidence interval:
##  0.4589373 0.5307412
## sample estimates:
##       cor 
## 0.4956858
\end{verbatim}

\begin{Shaded}
\begin{Highlighting}[]
\FunctionTok{ggplot}\NormalTok{(}\AttributeTok{data =}\NormalTok{ thetao\_8ab\_st200, }\FunctionTok{aes}\NormalTok{(}\AttributeTok{x =}\NormalTok{ temp\_degC, }\AttributeTok{y =}\NormalTok{ thetao)) }\SpecialCharTok{+} \FunctionTok{geom\_point}\NormalTok{(}\FunctionTok{aes}\NormalTok{(}\AttributeTok{color =}\NormalTok{ depth\_m)) }\SpecialCharTok{+} \FunctionTok{geom\_smooth}\NormalTok{(}\AttributeTok{method =} \StringTok{"lm"}\NormalTok{) }\SpecialCharTok{+} \FunctionTok{scale\_color\_viridis}\NormalTok{()}
\end{Highlighting}
\end{Shaded}

\begin{verbatim}
## `geom_smooth()` using formula = 'y ~ x'
\end{verbatim}

\begin{verbatim}
## Warning: Removed 918 rows containing non-finite values (`stat_smooth()`).
\end{verbatim}

\begin{verbatim}
## Warning: Removed 918 rows containing missing values (`geom_point()`).
\end{verbatim}

\includegraphics{si_sbt_in-situ-ctd-vs-modeled-temperature_files/figure-latex/unnamed-chunk-5-1.pdf}

At locations at depth under 200m (excluding the CTD locations with NA
depth), the correlation is much better: 0.50 {[}95CI 0.46-0.53{]}.
However, it seems that thetao predicts higher temperature when CTD is
within 10-15 degC.

Monthly average

\begin{Shaded}
\begin{Highlighting}[]
\CommentTok{\# correlation test aggregation by month}
\CommentTok{\# between monthly mean of ctd and monthly mean of isimip sbt}

\NormalTok{thetao\_df }\OtherTok{\textless{}{-}} \FunctionTok{read\_rds}\NormalTok{(}\FunctionTok{file.path}\NormalTok{(dir\_thetao, }\StringTok{"isimip\_sbt\_ices.rds"}\NormalTok{))}

\NormalTok{thetao\_df }\OtherTok{\textless{}{-}}\NormalTok{ thetao\_df }\SpecialCharTok{\%\textgreater{}\%} 
  \FunctionTok{mutate}\NormalTok{(}\AttributeTok{Month =} \FunctionTok{month}\NormalTok{(Date),}
         \AttributeTok{date\_match =} \FunctionTok{as.Date}\NormalTok{(}\FunctionTok{paste0}\NormalTok{(}\StringTok{"01"}\NormalTok{, }\StringTok{"{-}"}\NormalTok{, Month, }\StringTok{"{-}"}\NormalTok{, Year),}\AttributeTok{format =} \StringTok{"\%d{-}\%m{-}\%Y"}\NormalTok{), }
         \AttributeTok{Area\_27 =}\NormalTok{ IcesArea) }

\NormalTok{thetao\_join\_month }\OtherTok{\textless{}{-}}\NormalTok{ thetao\_join }\SpecialCharTok{\%\textgreater{}\%} 
  \FunctionTok{group\_by}\NormalTok{(Area\_27, date\_match) }\SpecialCharTok{\%\textgreater{}\%} 
  \FunctionTok{summarize}\NormalTok{(}\AttributeTok{temp\_degC =} \FunctionTok{mean}\NormalTok{(temp\_degC, }\AttributeTok{na.ram =}\NormalTok{ T))}
\end{Highlighting}
\end{Shaded}

\begin{verbatim}
## `summarise()` has grouped output by 'Area_27'. You can override using the
## `.groups` argument.
\end{verbatim}

\begin{Shaded}
\begin{Highlighting}[]
\NormalTok{thetao\_join\_month\_st200 }\OtherTok{\textless{}{-}}\NormalTok{ thetao\_join }\SpecialCharTok{\%\textgreater{}\%} 
  \FunctionTok{filter}\NormalTok{(depth\_m }\SpecialCharTok{\textless{}=} \DecValTok{200}\NormalTok{) }\SpecialCharTok{\%\textgreater{}\%} \CommentTok{\#only records with depth \textless{} 200m}
  \FunctionTok{group\_by}\NormalTok{(Area\_27, date\_match) }\SpecialCharTok{\%\textgreater{}\%} 
  \FunctionTok{summarize}\NormalTok{(}\AttributeTok{temp\_degC =} \FunctionTok{mean}\NormalTok{(temp\_degC, }\AttributeTok{na.ram =}\NormalTok{ T))}
\end{Highlighting}
\end{Shaded}

\begin{verbatim}
## `summarise()` has grouped output by 'Area_27'. You can override using the
## `.groups` argument.
\end{verbatim}

\begin{Shaded}
\begin{Highlighting}[]
\NormalTok{thetao\_join\_month }\OtherTok{\textless{}{-}} \FunctionTok{left\_join}\NormalTok{(thetao\_join\_month, thetao\_df)}
\end{Highlighting}
\end{Shaded}

\begin{verbatim}
## Joining, by = c("Area_27", "date_match")
\end{verbatim}

\begin{Shaded}
\begin{Highlighting}[]
\NormalTok{thetao\_join\_month\_st200 }\OtherTok{\textless{}{-}} \FunctionTok{left\_join}\NormalTok{(thetao\_join\_month\_st200, thetao\_df)}
\end{Highlighting}
\end{Shaded}

\begin{verbatim}
## Joining, by = c("Area_27", "date_match")
\end{verbatim}

\begin{Shaded}
\begin{Highlighting}[]
\CommentTok{\# Correlation test}
\FunctionTok{cor.test}\NormalTok{(thetao\_join\_month}\SpecialCharTok{$}\NormalTok{temp\_degC, thetao\_join\_month}\SpecialCharTok{$}\NormalTok{isimip\_sbt) }\CommentTok{\#0.86 [95CI 0.84{-}0.87]}
\end{Highlighting}
\end{Shaded}

\begin{verbatim}
## 
##  Pearson's product-moment correlation
## 
## data:  thetao_join_month$temp_degC and thetao_join_month$isimip_sbt
## t = 43.361, df = 669, p-value < 2.2e-16
## alternative hypothesis: true correlation is not equal to 0
## 95 percent confidence interval:
##  0.8375706 0.8774659
## sample estimates:
##       cor 
## 0.8588149
\end{verbatim}

\begin{Shaded}
\begin{Highlighting}[]
\NormalTok{thetao\_month\_4bc }\OtherTok{\textless{}{-}}\NormalTok{ thetao\_join\_month }\SpecialCharTok{\%\textgreater{}\%} \FunctionTok{filter}\NormalTok{(Area\_27 }\SpecialCharTok{==} \StringTok{"4bc"}\NormalTok{)}
\NormalTok{thetao\_month\_7a }\OtherTok{\textless{}{-}}\NormalTok{ thetao\_join\_month }\SpecialCharTok{\%\textgreater{}\%} \FunctionTok{filter}\NormalTok{(Area\_27 }\SpecialCharTok{==} \StringTok{"7a"}\NormalTok{)}
\NormalTok{thetao\_month\_8ab }\OtherTok{\textless{}{-}}\NormalTok{ thetao\_join\_month\_st200 }\SpecialCharTok{\%\textgreater{}\%} \FunctionTok{filter}\NormalTok{(Area\_27 }\SpecialCharTok{==} \StringTok{"8ab"}\NormalTok{)}

\FunctionTok{cor.test}\NormalTok{(thetao\_month\_4bc}\SpecialCharTok{$}\NormalTok{temp\_degC, thetao\_month\_4bc}\SpecialCharTok{$}\NormalTok{isimip\_sbt) }\CommentTok{\#0.85 [95CI 0.82{-}0.87]}
\end{Highlighting}
\end{Shaded}

\begin{verbatim}
## 
##  Pearson's product-moment correlation
## 
## data:  thetao_month_4bc$temp_degC and thetao_month_4bc$isimip_sbt
## t = 31.392, df = 383, p-value < 2.2e-16
## alternative hypothesis: true correlation is not equal to 0
## 95 percent confidence interval:
##  0.8180355 0.8743859
## sample estimates:
##       cor 
## 0.8486003
\end{verbatim}

\begin{Shaded}
\begin{Highlighting}[]
\FunctionTok{cor.test}\NormalTok{(thetao\_month\_7a}\SpecialCharTok{$}\NormalTok{temp\_degC, thetao\_month\_7a}\SpecialCharTok{$}\NormalTok{isimip\_sbt)   }\CommentTok{\#0.90 [95CI 0.86{-}0.93]}
\end{Highlighting}
\end{Shaded}

\begin{verbatim}
## 
##  Pearson's product-moment correlation
## 
## data:  thetao_month_7a$temp_degC and thetao_month_7a$isimip_sbt
## t = 23.771, df = 133, p-value < 2.2e-16
## alternative hypothesis: true correlation is not equal to 0
## 95 percent confidence interval:
##  0.8617456 0.9276503
## sample estimates:
##       cor 
## 0.8997071
\end{verbatim}

\begin{Shaded}
\begin{Highlighting}[]
\FunctionTok{cor.test}\NormalTok{(thetao\_month\_8ab}\SpecialCharTok{$}\NormalTok{temp\_degC, thetao\_month\_8ab}\SpecialCharTok{$}\NormalTok{isimip\_sbt) }\CommentTok{\#0.52 [95CI 0.36{-}0.65]}
\end{Highlighting}
\end{Shaded}

\begin{verbatim}
## 
##  Pearson's product-moment correlation
## 
## data:  thetao_month_8ab$temp_degC and thetao_month_8ab$isimip_sbt
## t = 6.1997, df = 104, p-value = 1.158e-08
## alternative hypothesis: true correlation is not equal to 0
## 95 percent confidence interval:
##  0.3648711 0.6461937
## sample estimates:
##       cor 
## 0.5194708
\end{verbatim}

\begin{Shaded}
\begin{Highlighting}[]
\CommentTok{\# plot}
\FunctionTok{ggplot}\NormalTok{(}\AttributeTok{data =}\NormalTok{ thetao\_join\_month, }\FunctionTok{aes}\NormalTok{(}\AttributeTok{x =}\NormalTok{ temp\_degC, }\AttributeTok{y =}\NormalTok{ isimip\_sbt)) }\SpecialCharTok{+} \FunctionTok{geom\_point}\NormalTok{() }\SpecialCharTok{+} \FunctionTok{geom\_smooth}\NormalTok{(}\AttributeTok{method =} \StringTok{"lm"}\NormalTok{)}
\end{Highlighting}
\end{Shaded}

\begin{verbatim}
## `geom_smooth()` using formula = 'y ~ x'
\end{verbatim}

\includegraphics{si_sbt_in-situ-ctd-vs-modeled-temperature_files/figure-latex/unnamed-chunk-6-1.pdf}

\begin{Shaded}
\begin{Highlighting}[]
\FunctionTok{ggplot}\NormalTok{(}\AttributeTok{data =}\NormalTok{ thetao\_month\_4bc, }\FunctionTok{aes}\NormalTok{(}\AttributeTok{x =}\NormalTok{ temp\_degC, }\AttributeTok{y =}\NormalTok{ isimip\_sbt)) }\SpecialCharTok{+} \FunctionTok{geom\_point}\NormalTok{() }\SpecialCharTok{+} \FunctionTok{geom\_smooth}\NormalTok{(}\AttributeTok{method =} \StringTok{"lm"}\NormalTok{)}
\end{Highlighting}
\end{Shaded}

\begin{verbatim}
## `geom_smooth()` using formula = 'y ~ x'
\end{verbatim}

\includegraphics{si_sbt_in-situ-ctd-vs-modeled-temperature_files/figure-latex/unnamed-chunk-6-2.pdf}

\begin{Shaded}
\begin{Highlighting}[]
\FunctionTok{ggplot}\NormalTok{(}\AttributeTok{data =}\NormalTok{ thetao\_month\_7a, }\FunctionTok{aes}\NormalTok{(}\AttributeTok{x =}\NormalTok{ temp\_degC, }\AttributeTok{y =}\NormalTok{ isimip\_sbt)) }\SpecialCharTok{+} \FunctionTok{geom\_point}\NormalTok{() }\SpecialCharTok{+} \FunctionTok{geom\_smooth}\NormalTok{(}\AttributeTok{method =} \StringTok{"lm"}\NormalTok{)}
\end{Highlighting}
\end{Shaded}

\begin{verbatim}
## `geom_smooth()` using formula = 'y ~ x'
\end{verbatim}

\includegraphics{si_sbt_in-situ-ctd-vs-modeled-temperature_files/figure-latex/unnamed-chunk-6-3.pdf}

\begin{Shaded}
\begin{Highlighting}[]
\FunctionTok{ggplot}\NormalTok{(}\AttributeTok{data =}\NormalTok{ thetao\_month\_8ab, }\FunctionTok{aes}\NormalTok{(}\AttributeTok{x =}\NormalTok{ temp\_degC, }\AttributeTok{y =}\NormalTok{ isimip\_sbt)) }\SpecialCharTok{+} \FunctionTok{geom\_point}\NormalTok{() }\SpecialCharTok{+} \FunctionTok{geom\_smooth}\NormalTok{(}\AttributeTok{method =} \StringTok{"lm"}\NormalTok{)}
\end{Highlighting}
\end{Shaded}

\begin{verbatim}
## `geom_smooth()` using formula = 'y ~ x'
\end{verbatim}

\includegraphics{si_sbt_in-situ-ctd-vs-modeled-temperature_files/figure-latex/unnamed-chunk-6-4.pdf}

Very high correlation of CTD and ISIMIP thetao at the North Sea 0.85
{[}95CI 0.82-0.87{]} and Irish Sea 0.90 {[}95CI 0.86-0.93{]} and
moderate correlation at the Bay of Biscay 0.52 {[}95CI 0.36-0.65{]}

\hypertarget{isimip-tob}{%
\paragraph{3.2. ISIMIP TOB}\label{isimip-tob}}

ISIMIP tob - potential sea bottom temperature. This is another dataset
from ISIMIP

\hypertarget{process-data-1}{%
\subparagraph{Process data}\label{process-data-1}}

\begin{Shaded}
\begin{Highlighting}[]
\CommentTok{\# load data}

\NormalTok{dir\_tob }\OtherTok{\textless{}{-}} \StringTok{"D:/OneDrive {-} UGent/data/Env/Sea Bottom Temperature\_ISIMIP3b\_MPI{-}ESM1{-}2HR\_0.5deg\_1850{-}2100"}

\NormalTok{file\_tob\_hist }\OtherTok{\textless{}{-}} \StringTok{"mpi{-}esm1{-}2{-}hr\_r1i1p1f1\_historical\_tob\_30arcmin\_global\_monthly\_1850\_2014.nc"}
\NormalTok{file\_tob\_ssp126 }\OtherTok{\textless{}{-}} \StringTok{"mpi{-}esm1{-}2{-}hr\_r1i1p1f1\_ssp126\_tob\_30arcmin\_global\_monthly\_2015\_2100.nc"}

\CommentTok{\# defnine bounding box}
\FunctionTok{st\_bbox}\NormalTok{(ices\_area)}
\end{Highlighting}
\end{Shaded}

\begin{verbatim}
##      xmin      ymin      xmax      ymax 
## -8.746612 43.314507 12.005942 57.500000
\end{verbatim}

\begin{Shaded}
\begin{Highlighting}[]
\NormalTok{bb }\OtherTok{=} \FunctionTok{st\_bbox}\NormalTok{(}\FunctionTok{c}\NormalTok{(}\AttributeTok{xmin=}\SpecialCharTok{{-}}\DecValTok{10}\NormalTok{, }\AttributeTok{ymin=}\DecValTok{42}\NormalTok{, }
               \AttributeTok{xmax=}\DecValTok{13}\NormalTok{, }\AttributeTok{ymax=}\DecValTok{58}\NormalTok{),}
             \AttributeTok{crs =} \FunctionTok{st\_crs}\NormalTok{(}\DecValTok{4326}\NormalTok{))}

\CommentTok{\# read and crop file by bounding box}
\NormalTok{tob\_hist }\OtherTok{\textless{}{-}} \FunctionTok{read\_ncdf}\NormalTok{(}\FunctionTok{file.path}\NormalTok{(dir\_tob, file\_tob\_hist), }\AttributeTok{proxy =} \ConstantTok{TRUE}\NormalTok{)}
\end{Highlighting}
\end{Shaded}

\begin{verbatim}
## no 'var' specified, using tob
\end{verbatim}

\begin{verbatim}
## other available variables:
##  time, time_bnds, lon, lat
\end{verbatim}

\begin{verbatim}
## No projection information found in nc file. 
##  Coordinate variable units found to be degrees, 
##  assuming WGS84 Lat/Lon.
\end{verbatim}

\begin{Shaded}
\begin{Highlighting}[]
\NormalTok{tob\_hist }\OtherTok{=}\NormalTok{ tob\_hist[bb]}

\NormalTok{tob\_ssp126 }\OtherTok{\textless{}{-}} \FunctionTok{read\_ncdf}\NormalTok{(}\FunctionTok{file.path}\NormalTok{(dir\_tob, file\_tob\_ssp126), }\AttributeTok{proxy =} \ConstantTok{TRUE}\NormalTok{)}
\end{Highlighting}
\end{Shaded}

\begin{verbatim}
## no 'var' specified, using tob
\end{verbatim}

\begin{verbatim}
## other available variables:
##  time, time_bnds, lon, lat
\end{verbatim}

\begin{verbatim}
## No projection information found in nc file. 
##  Coordinate variable units found to be degrees, 
##  assuming WGS84 Lat/Lon.
\end{verbatim}

\begin{Shaded}
\begin{Highlighting}[]
\NormalTok{tob\_ssp126 }\OtherTok{=}\NormalTok{ tob\_ssp126[bb]}

\CommentTok{\# merge tob\_hist and tob\_spp126}
\NormalTok{tob }\OtherTok{\textless{}{-}} \FunctionTok{c}\NormalTok{(tob\_hist, tob\_ssp126)}
\end{Highlighting}
\end{Shaded}

\begin{Shaded}
\begin{Highlighting}[]
\CommentTok{\# Extract values of isimip sbt at ctd points}
\CommentTok{\# Values of isimip sbt at certain year{-}month layer will be extracted at ctd point at that certain year{-}month}

\CommentTok{\# however the time stamp of isimip sbt and ctd are different}
\CommentTok{\# isimip sbt: date{-}time (around 15,16 of month)}
\CommentTok{\# ctd point: sample date}
\CommentTok{\# this difference leads to NA in extraction and conversion of time stamp is needed}
\CommentTok{\# time stamp (for matching) of both data will be converted to 1st day of each month (eg. 01{-}01{-}1990) }

\CommentTok{\# change time values isimip sbt}
\CommentTok{\# hist}
\NormalTok{year\_tob }\OtherTok{\textless{}{-}} \FunctionTok{year}\NormalTok{(}\FunctionTok{attr}\NormalTok{(tob, }\StringTok{"dimensions"}\NormalTok{)}\SpecialCharTok{$}\NormalTok{time}\SpecialCharTok{$}\NormalTok{values)}
\NormalTok{month\_tob }\OtherTok{\textless{}{-}} \FunctionTok{month}\NormalTok{(}\FunctionTok{attr}\NormalTok{(tob, }\StringTok{"dimensions"}\NormalTok{)}\SpecialCharTok{$}\NormalTok{time}\SpecialCharTok{$}\NormalTok{values)}
\NormalTok{date\_tob }\OtherTok{\textless{}{-}} \FunctionTok{as.Date}\NormalTok{(}\FunctionTok{paste0}\NormalTok{(}\StringTok{"01"}\NormalTok{, }\StringTok{"{-}"}\NormalTok{, month\_tob, }\StringTok{"{-}"}\NormalTok{, year\_tob),}\AttributeTok{format =} \StringTok{"\%d{-}\%m{-}\%Y"}\NormalTok{)}

\FunctionTok{attr}\NormalTok{(tob, }\StringTok{"dimensions"}\NormalTok{)}\SpecialCharTok{$}\NormalTok{time}\SpecialCharTok{$}\NormalTok{values }\OtherTok{\textless{}{-}}\NormalTok{ date\_tob}

\CommentTok{\# change time ctd\_test}
\NormalTok{ctd }\OtherTok{\textless{}{-}}\NormalTok{ ctd }\SpecialCharTok{\%\textgreater{}\%} \FunctionTok{mutate}\NormalTok{(}\AttributeTok{date\_match =} \FunctionTok{as.Date}\NormalTok{(}\FunctionTok{paste0}\NormalTok{(}\StringTok{"01"}\NormalTok{, }\StringTok{"{-}"}\NormalTok{, Month, }\StringTok{"{-}"}\NormalTok{, Year),}\AttributeTok{format =} \StringTok{"\%d{-}\%m{-}\%Y"}\NormalTok{))}

\NormalTok{tob\_extract }\OtherTok{\textless{}{-}} \FunctionTok{st\_extract}\NormalTok{(tob, ctd, }\AttributeTok{time\_column =} \StringTok{"date\_match"}\NormalTok{)}

\CommentTok{\# joint with ctd data for comparison}
\NormalTok{tob\_join }\OtherTok{\textless{}{-}} \FunctionTok{left\_join}\NormalTok{(}\FunctionTok{as.data.frame}\NormalTok{(ctd), }\FunctionTok{as.data.frame}\NormalTok{(tob\_extract))}
\end{Highlighting}
\end{Shaded}

\begin{verbatim}
## Joining, by = c("geom", "date_match")
\end{verbatim}

\begin{Shaded}
\begin{Highlighting}[]
\CommentTok{\# the rows of sbt\_join differs from ctd and sbt\_extract due to the lack of Cruise and Station information in joining}
\CommentTok{\# sbt\_join should be group\_by to have the correct rows}
\NormalTok{tob\_join}\SpecialCharTok{$}\NormalTok{tob }\OtherTok{\textless{}{-}} \FunctionTok{as.numeric}\NormalTok{(tob\_join}\SpecialCharTok{$}\NormalTok{attr) }\CommentTok{\# convert tob from units to numeric to summarize}
\NormalTok{tob\_join }\OtherTok{\textless{}{-}}\NormalTok{ tob\_join }\SpecialCharTok{\%\textgreater{}\%} \FunctionTok{group\_by}\NormalTok{(ID, Cruise, Station, Year, Month, Day, sample\_date, Area\_27, date\_match, geom, depth\_m) }\SpecialCharTok{\%\textgreater{}\%} \FunctionTok{summarize}\NormalTok{(}\AttributeTok{tob =} \FunctionTok{mean}\NormalTok{(tob), }\AttributeTok{temp\_degC =} \FunctionTok{mean}\NormalTok{(temp\_degC))}
\end{Highlighting}
\end{Shaded}

\begin{verbatim}
## `summarise()` has grouped output by 'ID', 'Cruise', 'Station', 'Year', 'Month',
## 'Day', 'sample_date', 'Area_27', 'date_match', 'geom'. You can override using
## the `.groups` argument.
\end{verbatim}

\hypertarget{correlation-test-1}{%
\subparagraph{Correlation test}\label{correlation-test-1}}

At CTD locations

\begin{Shaded}
\begin{Highlighting}[]
\CommentTok{\# Correlation test}

\CommentTok{\# all data}
\FunctionTok{cor.test}\NormalTok{(tob\_join}\SpecialCharTok{$}\NormalTok{temp\_degC, tob\_join}\SpecialCharTok{$}\NormalTok{tob) }\CommentTok{\#0.906 [95CI 0.905{-}0.908]}
\end{Highlighting}
\end{Shaded}

\begin{verbatim}
## 
##  Pearson's product-moment correlation
## 
## data:  tob_join$temp_degC and tob_join$tob
## t = 408.38, df = 36235, p-value < 2.2e-16
## alternative hypothesis: true correlation is not equal to 0
## 95 percent confidence interval:
##  0.9045174 0.9081932
## sample estimates:
##       cor 
## 0.9063724
\end{verbatim}

\begin{Shaded}
\begin{Highlighting}[]
\CommentTok{\# by ices division}
\NormalTok{tob\_4bc }\OtherTok{\textless{}{-}}\NormalTok{ tob\_join }\SpecialCharTok{\%\textgreater{}\%} \FunctionTok{filter}\NormalTok{(Area\_27 }\SpecialCharTok{==} \StringTok{"4bc"}\NormalTok{)}
\NormalTok{tob\_7a }\OtherTok{\textless{}{-}}\NormalTok{ tob\_join }\SpecialCharTok{\%\textgreater{}\%} \FunctionTok{filter}\NormalTok{(Area\_27 }\SpecialCharTok{==} \StringTok{"7a"}\NormalTok{)}
\NormalTok{tob\_8ab }\OtherTok{\textless{}{-}}\NormalTok{ tob\_join }\SpecialCharTok{\%\textgreater{}\%} \FunctionTok{filter}\NormalTok{(Area\_27 }\SpecialCharTok{==} \StringTok{"8ab"}\NormalTok{)}

\FunctionTok{cor.test}\NormalTok{(tob\_4bc}\SpecialCharTok{$}\NormalTok{temp\_degC, tob\_4bc}\SpecialCharTok{$}\NormalTok{tob) }\CommentTok{\#0.910 [95CI 0.907{-}0.901]}
\end{Highlighting}
\end{Shaded}

\begin{verbatim}
## 
##  Pearson's product-moment correlation
## 
## data:  tob_4bc$temp_degC and tob_4bc$tob
## t = 396.2, df = 32746, p-value < 2.2e-16
## alternative hypothesis: true correlation is not equal to 0
## 95 percent confidence interval:
##  0.9077256 0.9114648
## sample estimates:
##       cor 
## 0.9096136
\end{verbatim}

\begin{Shaded}
\begin{Highlighting}[]
\FunctionTok{cor.test}\NormalTok{(tob\_7a}\SpecialCharTok{$}\NormalTok{temp\_degC, tob\_7a}\SpecialCharTok{$}\NormalTok{tob)   }\CommentTok{\#0.85 [95CI 0.82{-}0.87]}
\end{Highlighting}
\end{Shaded}

\begin{verbatim}
## 
##  Pearson's product-moment correlation
## 
## data:  tob_7a$temp_degC and tob_7a$tob
## t = 35.666, df = 507, p-value < 2.2e-16
## alternative hypothesis: true correlation is not equal to 0
## 95 percent confidence interval:
##  0.8188514 0.8686575
## sample estimates:
##       cor 
## 0.8455846
\end{verbatim}

\begin{Shaded}
\begin{Highlighting}[]
\FunctionTok{cor.test}\NormalTok{(tob\_8ab}\SpecialCharTok{$}\NormalTok{temp\_degC, tob\_8ab}\SpecialCharTok{$}\NormalTok{tob) }\CommentTok{\#0.40 [95CI 0.36{-}0.42]}
\end{Highlighting}
\end{Shaded}

\begin{verbatim}
## 
##  Pearson's product-moment correlation
## 
## data:  tob_8ab$temp_degC and tob_8ab$tob
## t = 23.469, df = 2978, p-value < 2.2e-16
## alternative hypothesis: true correlation is not equal to 0
## 95 percent confidence interval:
##  0.3643369 0.4249530
## sample estimates:
##       cor 
## 0.3950749
\end{verbatim}

\begin{Shaded}
\begin{Highlighting}[]
\CommentTok{\# plot}
\FunctionTok{ggplot}\NormalTok{(}\AttributeTok{data =}\NormalTok{ tob\_join, }\FunctionTok{aes}\NormalTok{(}\AttributeTok{x =}\NormalTok{ temp\_degC, }\AttributeTok{y =}\NormalTok{ tob)) }\SpecialCharTok{+} \FunctionTok{geom\_point}\NormalTok{() }\SpecialCharTok{+} \FunctionTok{geom\_smooth}\NormalTok{(}\AttributeTok{method =} \StringTok{"lm"}\NormalTok{)}
\end{Highlighting}
\end{Shaded}

\begin{verbatim}
## `geom_smooth()` using formula = 'y ~ x'
\end{verbatim}

\begin{verbatim}
## Warning: Removed 11864 rows containing non-finite values (`stat_smooth()`).
\end{verbatim}

\begin{verbatim}
## Warning: Removed 11864 rows containing missing values (`geom_point()`).
\end{verbatim}

\includegraphics{si_sbt_in-situ-ctd-vs-modeled-temperature_files/figure-latex/unnamed-chunk-9-1.pdf}

\begin{Shaded}
\begin{Highlighting}[]
\FunctionTok{ggplot}\NormalTok{(}\AttributeTok{data =}\NormalTok{ tob\_4bc, }\FunctionTok{aes}\NormalTok{(}\AttributeTok{x =}\NormalTok{ temp\_degC, }\AttributeTok{y =}\NormalTok{ tob)) }\SpecialCharTok{+} \FunctionTok{geom\_point}\NormalTok{() }\SpecialCharTok{+} \FunctionTok{geom\_smooth}\NormalTok{(}\AttributeTok{method =} \StringTok{"lm"}\NormalTok{)}
\end{Highlighting}
\end{Shaded}

\begin{verbatim}
## `geom_smooth()` using formula = 'y ~ x'
\end{verbatim}

\begin{verbatim}
## Warning: Removed 8300 rows containing non-finite values (`stat_smooth()`).
\end{verbatim}

\begin{verbatim}
## Warning: Removed 8300 rows containing missing values (`geom_point()`).
\end{verbatim}

\includegraphics{si_sbt_in-situ-ctd-vs-modeled-temperature_files/figure-latex/unnamed-chunk-9-2.pdf}

\begin{Shaded}
\begin{Highlighting}[]
\FunctionTok{ggplot}\NormalTok{(}\AttributeTok{data =}\NormalTok{ tob\_7a, }\FunctionTok{aes}\NormalTok{(}\AttributeTok{x =}\NormalTok{ temp\_degC, }\AttributeTok{y =}\NormalTok{ tob)) }\SpecialCharTok{+} \FunctionTok{geom\_point}\NormalTok{() }\SpecialCharTok{+} \FunctionTok{geom\_smooth}\NormalTok{(}\AttributeTok{method =} \StringTok{"lm"}\NormalTok{)}
\end{Highlighting}
\end{Shaded}

\begin{verbatim}
## `geom_smooth()` using formula = 'y ~ x'
\end{verbatim}

\begin{verbatim}
## Warning: Removed 2103 rows containing non-finite values (`stat_smooth()`).
\end{verbatim}

\begin{verbatim}
## Warning: Removed 2103 rows containing missing values (`geom_point()`).
\end{verbatim}

\includegraphics{si_sbt_in-situ-ctd-vs-modeled-temperature_files/figure-latex/unnamed-chunk-9-3.pdf}

\begin{Shaded}
\begin{Highlighting}[]
\FunctionTok{ggplot}\NormalTok{(}\AttributeTok{data =}\NormalTok{ tob\_8ab, }\FunctionTok{aes}\NormalTok{(}\AttributeTok{x =}\NormalTok{ temp\_degC, }\AttributeTok{y =}\NormalTok{ tob)) }\SpecialCharTok{+} \FunctionTok{geom\_point}\NormalTok{() }\SpecialCharTok{+} \FunctionTok{geom\_smooth}\NormalTok{(}\AttributeTok{method =} \StringTok{"lm"}\NormalTok{)}
\end{Highlighting}
\end{Shaded}

\begin{verbatim}
## `geom_smooth()` using formula = 'y ~ x'
\end{verbatim}

\begin{verbatim}
## Warning: Removed 1461 rows containing non-finite values (`stat_smooth()`).
\end{verbatim}

\begin{verbatim}
## Warning: Removed 1461 rows containing missing values (`geom_point()`).
\end{verbatim}

\includegraphics{si_sbt_in-situ-ctd-vs-modeled-temperature_files/figure-latex/unnamed-chunk-9-4.pdf}

Similar to ISIMIP thetao, CTD is highly correlated with ISIMIP tob in
the North Sea and the Irish Sea but weakly correlated in the Bay of
Biscay 0.40 {[}95CI 0.36-0.42{]}.

Another test for CTD at depth under 200m in the Bay of Biscay is needed

\begin{Shaded}
\begin{Highlighting}[]
\NormalTok{tob\_8ab\_st200 }\OtherTok{\textless{}{-}}\NormalTok{ thetao\_8ab }\SpecialCharTok{\%\textgreater{}\%} \FunctionTok{filter}\NormalTok{(depth\_m }\SpecialCharTok{\textless{}} \DecValTok{200}\NormalTok{)}

\FunctionTok{cor.test}\NormalTok{(tob\_8ab\_st200}\SpecialCharTok{$}\NormalTok{temp\_degC, tob\_8ab\_st200}\SpecialCharTok{$}\NormalTok{thetao) }\CommentTok{\#0.50 [95CI 0.46{-}0.53]}
\end{Highlighting}
\end{Shaded}

\begin{verbatim}
## 
##  Pearson's product-moment correlation
## 
## data:  tob_8ab_st200$temp_degC and tob_8ab_st200$thetao
## t = 23.504, df = 1696, p-value < 2.2e-16
## alternative hypothesis: true correlation is not equal to 0
## 95 percent confidence interval:
##  0.4589373 0.5307412
## sample estimates:
##       cor 
## 0.4956858
\end{verbatim}

\begin{Shaded}
\begin{Highlighting}[]
\FunctionTok{ggplot}\NormalTok{(}\AttributeTok{data =}\NormalTok{ tob\_8ab\_st200, }\FunctionTok{aes}\NormalTok{(}\AttributeTok{x =}\NormalTok{ temp\_degC, }\AttributeTok{y =}\NormalTok{ thetao)) }\SpecialCharTok{+} \FunctionTok{geom\_point}\NormalTok{(}\FunctionTok{aes}\NormalTok{(}\AttributeTok{color =}\NormalTok{ depth\_m)) }\SpecialCharTok{+} \FunctionTok{geom\_smooth}\NormalTok{(}\AttributeTok{method =} \StringTok{"lm"}\NormalTok{) }\SpecialCharTok{+} \FunctionTok{scale\_color\_viridis}\NormalTok{()}
\end{Highlighting}
\end{Shaded}

\begin{verbatim}
## `geom_smooth()` using formula = 'y ~ x'
\end{verbatim}

\begin{verbatim}
## Warning: Removed 918 rows containing non-finite values (`stat_smooth()`).
\end{verbatim}

\begin{verbatim}
## Warning: Removed 918 rows containing missing values (`geom_point()`).
\end{verbatim}

\includegraphics{si_sbt_in-situ-ctd-vs-modeled-temperature_files/figure-latex/unnamed-chunk-10-1.pdf}

Monthly average

\begin{Shaded}
\begin{Highlighting}[]
\CommentTok{\# correlation test aggregation by month}
\CommentTok{\# between monthly mean of ctd and monthly mean of isimip sbt}

\NormalTok{tob\_df }\OtherTok{\textless{}{-}} \FunctionTok{read\_rds}\NormalTok{(}\FunctionTok{file.path}\NormalTok{(dir\_tob, }\StringTok{"isimip\_sbt\_ices\_hist\_ssp126.rds"}\NormalTok{))}

\NormalTok{tob\_df }\OtherTok{\textless{}{-}}\NormalTok{ tob\_df }\SpecialCharTok{\%\textgreater{}\%}
  \FunctionTok{filter}\NormalTok{(IcesArea }\SpecialCharTok{\%in\%} \FunctionTok{c}\NormalTok{(}\StringTok{"4bc"}\NormalTok{, }\StringTok{"7a"}\NormalTok{, }\StringTok{"8ab"}\NormalTok{)) }\SpecialCharTok{\%\textgreater{}\%}
  \FunctionTok{mutate}\NormalTok{(}\AttributeTok{Month =} \FunctionTok{month}\NormalTok{(Date),}
         \AttributeTok{date\_match =} \FunctionTok{as.Date}\NormalTok{(}\FunctionTok{paste0}\NormalTok{(}\StringTok{"01"}\NormalTok{, }\StringTok{"{-}"}\NormalTok{, Month, }\StringTok{"{-}"}\NormalTok{, Year),}\AttributeTok{format =} \StringTok{"\%d{-}\%m{-}\%Y"}\NormalTok{), }
         \AttributeTok{Area\_27 =}\NormalTok{ IcesArea,}
         \AttributeTok{isimip\_sbt =} \FunctionTok{as.numeric}\NormalTok{(isimip\_sbt)) }

\NormalTok{tob\_join\_month }\OtherTok{\textless{}{-}}\NormalTok{ tob\_join }\SpecialCharTok{\%\textgreater{}\%} 
  \FunctionTok{group\_by}\NormalTok{(Area\_27, date\_match) }\SpecialCharTok{\%\textgreater{}\%} 
  \FunctionTok{summarize}\NormalTok{(}\AttributeTok{temp\_degC =} \FunctionTok{mean}\NormalTok{(temp\_degC, }\AttributeTok{na.ram =}\NormalTok{ T))}
\end{Highlighting}
\end{Shaded}

\begin{verbatim}
## `summarise()` has grouped output by 'Area_27'. You can override using the
## `.groups` argument.
\end{verbatim}

\begin{Shaded}
\begin{Highlighting}[]
\NormalTok{tob\_join\_month\_st200 }\OtherTok{\textless{}{-}}\NormalTok{ tob\_join }\SpecialCharTok{\%\textgreater{}\%} 
  \FunctionTok{filter}\NormalTok{(depth\_m }\SpecialCharTok{\textless{}=} \DecValTok{200}\NormalTok{) }\SpecialCharTok{\%\textgreater{}\%} \CommentTok{\#only records with depth \textless{} 200m}
  \FunctionTok{group\_by}\NormalTok{(Area\_27, date\_match) }\SpecialCharTok{\%\textgreater{}\%} 
  \FunctionTok{summarize}\NormalTok{(}\AttributeTok{temp\_degC =} \FunctionTok{mean}\NormalTok{(temp\_degC, }\AttributeTok{na.ram =}\NormalTok{ T))}
\end{Highlighting}
\end{Shaded}

\begin{verbatim}
## `summarise()` has grouped output by 'Area_27'. You can override using the
## `.groups` argument.
\end{verbatim}

\begin{Shaded}
\begin{Highlighting}[]
\NormalTok{tob\_join\_month }\OtherTok{\textless{}{-}} \FunctionTok{left\_join}\NormalTok{(tob\_join\_month, tob\_df)}
\end{Highlighting}
\end{Shaded}

\begin{verbatim}
## Joining, by = c("Area_27", "date_match")
\end{verbatim}

\begin{Shaded}
\begin{Highlighting}[]
\NormalTok{tob\_join\_month\_st200 }\OtherTok{\textless{}{-}} \FunctionTok{left\_join}\NormalTok{(tob\_join\_month\_st200, tob\_df)}
\end{Highlighting}
\end{Shaded}

\begin{verbatim}
## Joining, by = c("Area_27", "date_match")
\end{verbatim}

\begin{Shaded}
\begin{Highlighting}[]
\CommentTok{\# Correlation test}
\FunctionTok{cor.test}\NormalTok{(tob\_join\_month}\SpecialCharTok{$}\NormalTok{temp\_degC, tob\_join\_month}\SpecialCharTok{$}\NormalTok{isimip\_sbt) }\CommentTok{\#0.85 [95CI 0.83{-}0.87]}
\end{Highlighting}
\end{Shaded}

\begin{verbatim}
## 
##  Pearson's product-moment correlation
## 
## data:  tob_join_month$temp_degC and tob_join_month$isimip_sbt
## t = 41.923, df = 669, p-value < 2.2e-16
## alternative hypothesis: true correlation is not equal to 0
## 95 percent confidence interval:
##  0.8287555 0.8706636
## sample estimates:
##       cor 
## 0.8510593
\end{verbatim}

\begin{Shaded}
\begin{Highlighting}[]
\NormalTok{tob\_month\_4bc }\OtherTok{\textless{}{-}}\NormalTok{ tob\_join\_month }\SpecialCharTok{\%\textgreater{}\%} \FunctionTok{filter}\NormalTok{(Area\_27 }\SpecialCharTok{==} \StringTok{"4bc"}\NormalTok{)}
\NormalTok{tob\_month\_7a }\OtherTok{\textless{}{-}}\NormalTok{ tob\_join\_month }\SpecialCharTok{\%\textgreater{}\%} \FunctionTok{filter}\NormalTok{(Area\_27 }\SpecialCharTok{==} \StringTok{"7a"}\NormalTok{)}
\NormalTok{tob\_month\_8ab }\OtherTok{\textless{}{-}}\NormalTok{ tob\_join\_month\_st200 }\SpecialCharTok{\%\textgreater{}\%} \FunctionTok{filter}\NormalTok{(Area\_27 }\SpecialCharTok{==} \StringTok{"8ab"}\NormalTok{)}

\FunctionTok{cor.test}\NormalTok{(tob\_month\_4bc}\SpecialCharTok{$}\NormalTok{temp\_degC, tob\_month\_4bc}\SpecialCharTok{$}\NormalTok{isimip\_sbt) }\CommentTok{\#0.84 [95CI 0.81{-}0.87]}
\end{Highlighting}
\end{Shaded}

\begin{verbatim}
## 
##  Pearson's product-moment correlation
## 
## data:  tob_month_4bc$temp_degC and tob_month_4bc$isimip_sbt
## t = 30.226, df = 383, p-value < 2.2e-16
## alternative hypothesis: true correlation is not equal to 0
## 95 percent confidence interval:
##  0.8071860 0.8666498
## sample estimates:
##       cor 
## 0.8394123
\end{verbatim}

\begin{Shaded}
\begin{Highlighting}[]
\FunctionTok{cor.test}\NormalTok{(tob\_month\_7a}\SpecialCharTok{$}\NormalTok{temp\_degC, tob\_month\_7a}\SpecialCharTok{$}\NormalTok{isimip\_sbt)   }\CommentTok{\#0.91 [95CI 0.88{-}0.94]}
\end{Highlighting}
\end{Shaded}

\begin{verbatim}
## 
##  Pearson's product-moment correlation
## 
## data:  tob_month_7a$temp_degC and tob_month_7a$isimip_sbt
## t = 25.783, df = 133, p-value < 2.2e-16
## alternative hypothesis: true correlation is not equal to 0
## 95 percent confidence interval:
##  0.8795368 0.9372465
## sample estimates:
##      cor 
## 0.912842
\end{verbatim}

\begin{Shaded}
\begin{Highlighting}[]
\FunctionTok{cor.test}\NormalTok{(tob\_month\_8ab}\SpecialCharTok{$}\NormalTok{temp\_degC, tob\_month\_8ab}\SpecialCharTok{$}\NormalTok{isimip\_sbt) }\CommentTok{\#0.43 [95CI 0.26{-}0.57]}
\end{Highlighting}
\end{Shaded}

\begin{verbatim}
## 
##  Pearson's product-moment correlation
## 
## data:  tob_month_8ab$temp_degC and tob_month_8ab$isimip_sbt
## t = 4.8556, df = 104, p-value = 4.247e-06
## alternative hypothesis: true correlation is not equal to 0
## 95 percent confidence interval:
##  0.2604951 0.5736066
## sample estimates:
##      cor 
## 0.429889
\end{verbatim}

\begin{Shaded}
\begin{Highlighting}[]
\CommentTok{\# plot}
\FunctionTok{ggplot}\NormalTok{(}\AttributeTok{data =}\NormalTok{ tob\_join\_month, }\FunctionTok{aes}\NormalTok{(}\AttributeTok{x =}\NormalTok{ temp\_degC, }\AttributeTok{y =}\NormalTok{ isimip\_sbt)) }\SpecialCharTok{+} \FunctionTok{geom\_point}\NormalTok{() }\SpecialCharTok{+} \FunctionTok{geom\_smooth}\NormalTok{(}\AttributeTok{method =} \StringTok{"lm"}\NormalTok{)}
\end{Highlighting}
\end{Shaded}

\begin{verbatim}
## `geom_smooth()` using formula = 'y ~ x'
\end{verbatim}

\includegraphics{si_sbt_in-situ-ctd-vs-modeled-temperature_files/figure-latex/unnamed-chunk-11-1.pdf}

\begin{Shaded}
\begin{Highlighting}[]
\FunctionTok{ggplot}\NormalTok{(}\AttributeTok{data =}\NormalTok{ tob\_month\_4bc, }\FunctionTok{aes}\NormalTok{(}\AttributeTok{x =}\NormalTok{ temp\_degC, }\AttributeTok{y =}\NormalTok{ isimip\_sbt)) }\SpecialCharTok{+} \FunctionTok{geom\_point}\NormalTok{() }\SpecialCharTok{+} \FunctionTok{geom\_smooth}\NormalTok{(}\AttributeTok{method =} \StringTok{"lm"}\NormalTok{)}
\end{Highlighting}
\end{Shaded}

\begin{verbatim}
## `geom_smooth()` using formula = 'y ~ x'
\end{verbatim}

\includegraphics{si_sbt_in-situ-ctd-vs-modeled-temperature_files/figure-latex/unnamed-chunk-11-2.pdf}

\begin{Shaded}
\begin{Highlighting}[]
\FunctionTok{ggplot}\NormalTok{(}\AttributeTok{data =}\NormalTok{ tob\_month\_7a, }\FunctionTok{aes}\NormalTok{(}\AttributeTok{x =}\NormalTok{ temp\_degC, }\AttributeTok{y =}\NormalTok{ isimip\_sbt)) }\SpecialCharTok{+} \FunctionTok{geom\_point}\NormalTok{() }\SpecialCharTok{+} \FunctionTok{geom\_smooth}\NormalTok{(}\AttributeTok{method =} \StringTok{"lm"}\NormalTok{)}
\end{Highlighting}
\end{Shaded}

\begin{verbatim}
## `geom_smooth()` using formula = 'y ~ x'
\end{verbatim}

\includegraphics{si_sbt_in-situ-ctd-vs-modeled-temperature_files/figure-latex/unnamed-chunk-11-3.pdf}

\begin{Shaded}
\begin{Highlighting}[]
\FunctionTok{ggplot}\NormalTok{(}\AttributeTok{data =}\NormalTok{ tob\_month\_8ab, }\FunctionTok{aes}\NormalTok{(}\AttributeTok{x =}\NormalTok{ temp\_degC, }\AttributeTok{y =}\NormalTok{ isimip\_sbt)) }\SpecialCharTok{+} \FunctionTok{geom\_point}\NormalTok{() }\SpecialCharTok{+} \FunctionTok{geom\_smooth}\NormalTok{(}\AttributeTok{method =} \StringTok{"lm"}\NormalTok{)}
\end{Highlighting}
\end{Shaded}

\begin{verbatim}
## `geom_smooth()` using formula = 'y ~ x'
\end{verbatim}

\includegraphics{si_sbt_in-situ-ctd-vs-modeled-temperature_files/figure-latex/unnamed-chunk-11-4.pdf}

The results are similar to that of monthly average ISIMIP thetao.
However, ISIMIP tob has lower correlation with CTD in the Bay of Biscay
0.43 {[}95CI 0.26-0.57{]}, compared to that of ISIMIP thetao 0.52
{[}95CI 0.36-0.65{]}

\hypertarget{hadisst}{%
\paragraph{3.3. HADISST}\label{hadisst}}

\hypertarget{correlation-test---monthly-average}{%
\subparagraph{Correlation test - monthly
average}\label{correlation-test---monthly-average}}

\begin{Shaded}
\begin{Highlighting}[]
\CommentTok{\# Correlation test}

\NormalTok{dir\_hadisst }\OtherTok{\textless{}{-}} \StringTok{"D:/OneDrive {-} UGent/data/Env/Sea Surface Temperature\_HadISST\_1deg\_1870{-}2021"}
\NormalTok{file\_hadisst }\OtherTok{\textless{}{-}} \StringTok{"haidsst\_ices.rds"}

\NormalTok{hadisst\_df }\OtherTok{\textless{}{-}} \FunctionTok{read\_rds}\NormalTok{(}\FunctionTok{file.path}\NormalTok{(dir\_hadisst, file\_hadisst))}

\NormalTok{hadisst\_df }\OtherTok{\textless{}{-}}\NormalTok{ hadisst\_df }\SpecialCharTok{\%\textgreater{}\%} \FunctionTok{filter}\NormalTok{(IcesArea }\SpecialCharTok{\%in\%} \FunctionTok{c}\NormalTok{(}\StringTok{"4bc"}\NormalTok{, }\StringTok{"7a"}\NormalTok{, }\StringTok{"8ab"}\NormalTok{))}
\NormalTok{hadisst\_df }\OtherTok{\textless{}{-}}\NormalTok{ hadisst\_df }\SpecialCharTok{\%\textgreater{}\%} \FunctionTok{mutate}\NormalTok{(}\AttributeTok{Area\_27 =}\NormalTok{ IcesArea,}
                              \AttributeTok{Month =} \FunctionTok{month}\NormalTok{(Date), }
                              \AttributeTok{date\_match =} \FunctionTok{as.Date}\NormalTok{(}\FunctionTok{paste0}\NormalTok{(}\StringTok{"01"}\NormalTok{, }\StringTok{"{-}"}\NormalTok{, Month, }\StringTok{"{-}"}\NormalTok{, Year),}\AttributeTok{format =} \StringTok{"\%d{-}\%m{-}\%Y"}\NormalTok{))}

\CommentTok{\# join }
\NormalTok{hadisst\_join }\OtherTok{\textless{}{-}} \FunctionTok{left\_join}\NormalTok{(}\FunctionTok{as.data.frame}\NormalTok{(ctd), hadisst\_df)}
\end{Highlighting}
\end{Shaded}

\begin{verbatim}
## Joining, by = c("Year", "Month", "Area_27", "date_match")
\end{verbatim}

\begin{Shaded}
\begin{Highlighting}[]
\NormalTok{hadisst\_join\_month }\OtherTok{\textless{}{-}}\NormalTok{ hadisst\_join }\SpecialCharTok{\%\textgreater{}\%} 
  \FunctionTok{group\_by}\NormalTok{(Area\_27, date\_match) }\SpecialCharTok{\%\textgreater{}\%} 
  \FunctionTok{summarize}\NormalTok{(}\AttributeTok{temp\_degC =} \FunctionTok{mean}\NormalTok{(temp\_degC, }\AttributeTok{na.rm =}\NormalTok{ T),}
            \AttributeTok{hadisst\_degC =} \FunctionTok{mean}\NormalTok{(hadisst\_degC, }\AttributeTok{na.rm =}\NormalTok{ T))}
\end{Highlighting}
\end{Shaded}

\begin{verbatim}
## `summarise()` has grouped output by 'Area_27'. You can override using the
## `.groups` argument.
\end{verbatim}

\begin{Shaded}
\begin{Highlighting}[]
\NormalTok{hadisst\_join\_month\_st200 }\OtherTok{\textless{}{-}}\NormalTok{ hadisst\_join }\SpecialCharTok{\%\textgreater{}\%} 
  \FunctionTok{filter}\NormalTok{(depth\_m }\SpecialCharTok{\textless{}=} \DecValTok{200}\NormalTok{) }\SpecialCharTok{\%\textgreater{}\%} \CommentTok{\#only records with depth \textless{} 200m}
  \FunctionTok{group\_by}\NormalTok{(Area\_27, date\_match) }\SpecialCharTok{\%\textgreater{}\%} 
  \FunctionTok{summarize}\NormalTok{(}\AttributeTok{temp\_degC =} \FunctionTok{mean}\NormalTok{(temp\_degC, }\AttributeTok{na.ram =}\NormalTok{ T),}
            \AttributeTok{hadisst\_degC =} \FunctionTok{mean}\NormalTok{(hadisst\_degC, }\AttributeTok{na.rm =}\NormalTok{ T))}
\end{Highlighting}
\end{Shaded}

\begin{verbatim}
## `summarise()` has grouped output by 'Area_27'. You can override using the
## `.groups` argument.
\end{verbatim}

\begin{Shaded}
\begin{Highlighting}[]
\CommentTok{\# Correlation test}
\FunctionTok{cor.test}\NormalTok{(hadisst\_join\_month}\SpecialCharTok{$}\NormalTok{temp\_degC, hadisst\_join\_month}\SpecialCharTok{$}\NormalTok{hadisst\_degC) }\CommentTok{\#0.83 [95CI 0.81{-}0.86]}
\end{Highlighting}
\end{Shaded}

\begin{verbatim}
## 
##  Pearson's product-moment correlation
## 
## data:  hadisst_join_month$temp_degC and hadisst_join_month$hadisst_degC
## t = 39.148, df = 669, p-value < 2.2e-16
## alternative hypothesis: true correlation is not equal to 0
## 95 percent confidence interval:
##  0.8097919 0.8559754
## sample estimates:
##       cor 
## 0.8343419
\end{verbatim}

\begin{Shaded}
\begin{Highlighting}[]
\NormalTok{hadisst\_month\_4bc }\OtherTok{\textless{}{-}}\NormalTok{ hadisst\_join\_month }\SpecialCharTok{\%\textgreater{}\%} \FunctionTok{filter}\NormalTok{(Area\_27 }\SpecialCharTok{==} \StringTok{"4bc"}\NormalTok{)}
\NormalTok{hadisst\_month\_7a }\OtherTok{\textless{}{-}}\NormalTok{ hadisst\_join\_month }\SpecialCharTok{\%\textgreater{}\%} \FunctionTok{filter}\NormalTok{(Area\_27 }\SpecialCharTok{==} \StringTok{"7a"}\NormalTok{)}
\NormalTok{hadisst\_month\_8ab }\OtherTok{\textless{}{-}}\NormalTok{ hadisst\_join\_month\_st200 }\SpecialCharTok{\%\textgreater{}\%} \FunctionTok{filter}\NormalTok{(Area\_27 }\SpecialCharTok{==} \StringTok{"8ab"}\NormalTok{)}

\FunctionTok{cor.test}\NormalTok{(hadisst\_month\_4bc}\SpecialCharTok{$}\NormalTok{temp\_degC, hadisst\_month\_4bc}\SpecialCharTok{$}\NormalTok{hadisst\_degC) }\CommentTok{\#0.85 [95CI 0.82{-}0.88]}
\end{Highlighting}
\end{Shaded}

\begin{verbatim}
## 
##  Pearson's product-moment correlation
## 
## data:  hadisst_month_4bc$temp_degC and hadisst_month_4bc$hadisst_degC
## t = 31.65, df = 383, p-value < 2.2e-16
## alternative hypothesis: true correlation is not equal to 0
## 95 percent confidence interval:
##  0.8203213 0.8760122
## sample estimates:
##       cor 
## 0.8505338
\end{verbatim}

\begin{Shaded}
\begin{Highlighting}[]
\FunctionTok{cor.test}\NormalTok{(hadisst\_month\_7a}\SpecialCharTok{$}\NormalTok{temp\_degC, hadisst\_month\_7a}\SpecialCharTok{$}\NormalTok{hadisst\_degC)   }\CommentTok{\#0.93 [95CI 0.90{-}0.95]}
\end{Highlighting}
\end{Shaded}

\begin{verbatim}
## 
##  Pearson's product-moment correlation
## 
## data:  hadisst_month_7a$temp_degC and hadisst_month_7a$hadisst_degC
## t = 28.746, df = 133, p-value < 2.2e-16
## alternative hypothesis: true correlation is not equal to 0
## 95 percent confidence interval:
##  0.9003185 0.9483460
## sample estimates:
##       cor 
## 0.9280978
\end{verbatim}

\begin{Shaded}
\begin{Highlighting}[]
\FunctionTok{cor.test}\NormalTok{(hadisst\_month\_8ab}\SpecialCharTok{$}\NormalTok{temp\_degC, hadisst\_month\_8ab}\SpecialCharTok{$}\NormalTok{hadisst\_degC) }\CommentTok{\#0.56 [95CI 0.41{-}0.68]}
\end{Highlighting}
\end{Shaded}

\begin{verbatim}
## 
##  Pearson's product-moment correlation
## 
## data:  hadisst_month_8ab$temp_degC and hadisst_month_8ab$hadisst_degC
## t = 6.8509, df = 104, p-value = 5.322e-10
## alternative hypothesis: true correlation is not equal to 0
## 95 percent confidence interval:
##  0.4105545 0.6764391
## sample estimates:
##       cor 
## 0.5576382
\end{verbatim}

\begin{Shaded}
\begin{Highlighting}[]
\FunctionTok{library}\NormalTok{(units)}
\end{Highlighting}
\end{Shaded}

\begin{verbatim}
## udunits database from C:/Users/tbui/AppData/Local/R/win-library/4.2/units/share/udunits/udunits2.xml
\end{verbatim}

\begin{Shaded}
\begin{Highlighting}[]
\FunctionTok{ggplot}\NormalTok{(}\AttributeTok{data =}\NormalTok{ hadisst\_join\_month, }\FunctionTok{aes}\NormalTok{(}\AttributeTok{x =}\NormalTok{ temp\_degC, }\AttributeTok{y =}\NormalTok{ hadisst\_degC)) }\SpecialCharTok{+} \FunctionTok{geom\_point}\NormalTok{() }\SpecialCharTok{+} \FunctionTok{geom\_smooth}\NormalTok{(}\AttributeTok{method =} \StringTok{"lm"}\NormalTok{)}
\end{Highlighting}
\end{Shaded}

\begin{verbatim}
## `geom_smooth()` using formula = 'y ~ x'
\end{verbatim}

\includegraphics{si_sbt_in-situ-ctd-vs-modeled-temperature_files/figure-latex/unnamed-chunk-12-1.pdf}

\begin{Shaded}
\begin{Highlighting}[]
\FunctionTok{ggplot}\NormalTok{(}\AttributeTok{data =}\NormalTok{ hadisst\_month\_4bc, }\FunctionTok{aes}\NormalTok{(}\AttributeTok{x =}\NormalTok{ temp\_degC, }\AttributeTok{y =}\NormalTok{ hadisst\_degC)) }\SpecialCharTok{+} \FunctionTok{geom\_point}\NormalTok{() }\SpecialCharTok{+} \FunctionTok{geom\_smooth}\NormalTok{(}\AttributeTok{method =} \StringTok{"lm"}\NormalTok{)}
\end{Highlighting}
\end{Shaded}

\begin{verbatim}
## `geom_smooth()` using formula = 'y ~ x'
\end{verbatim}

\includegraphics{si_sbt_in-situ-ctd-vs-modeled-temperature_files/figure-latex/unnamed-chunk-12-2.pdf}

\begin{Shaded}
\begin{Highlighting}[]
\FunctionTok{ggplot}\NormalTok{(}\AttributeTok{data =}\NormalTok{ hadisst\_month\_7a, }\FunctionTok{aes}\NormalTok{(}\AttributeTok{x =}\NormalTok{ temp\_degC, }\AttributeTok{y =}\NormalTok{ hadisst\_degC)) }\SpecialCharTok{+} \FunctionTok{geom\_point}\NormalTok{() }\SpecialCharTok{+} \FunctionTok{geom\_smooth}\NormalTok{(}\AttributeTok{method =} \StringTok{"lm"}\NormalTok{)}
\end{Highlighting}
\end{Shaded}

\begin{verbatim}
## `geom_smooth()` using formula = 'y ~ x'
\end{verbatim}

\includegraphics{si_sbt_in-situ-ctd-vs-modeled-temperature_files/figure-latex/unnamed-chunk-12-3.pdf}

\begin{Shaded}
\begin{Highlighting}[]
\FunctionTok{ggplot}\NormalTok{(}\AttributeTok{data =}\NormalTok{ hadisst\_month\_8ab, }\FunctionTok{aes}\NormalTok{(}\AttributeTok{x =}\NormalTok{ temp\_degC, }\AttributeTok{y =}\NormalTok{ hadisst\_degC)) }\SpecialCharTok{+} \FunctionTok{geom\_point}\NormalTok{() }\SpecialCharTok{+} \FunctionTok{geom\_smooth}\NormalTok{(}\AttributeTok{method =} \StringTok{"lm"}\NormalTok{)}
\end{Highlighting}
\end{Shaded}

\begin{verbatim}
## `geom_smooth()` using formula = 'y ~ x'
\end{verbatim}

\includegraphics{si_sbt_in-situ-ctd-vs-modeled-temperature_files/figure-latex/unnamed-chunk-12-4.pdf}

HADISST showed higher correlation to CTD in the Bay of Biscay, compared
to ISIMIP thetao and ISIMIP tob.

However, besides good correlation, the temperature should have a good
match in term of temperature range. Below we will explore that.

\hypertarget{cmems-north-west-shelf}{%
\paragraph{3.4. CMEMS NORTH WEST SHELF}\label{cmems-north-west-shelf}}

\hypertarget{process-data-2}{%
\subparagraph{Process data}\label{process-data-2}}

\begin{Shaded}
\begin{Highlighting}[]
\CommentTok{\# load data}

\NormalTok{dir\_nws }\OtherTok{\textless{}{-}} \StringTok{"D:/OneDrive {-} UGent/data/Env/Sea Bottom Temperature\_CMEMS\_Atlantic{-} European North West Shelf{-} Ocean Physics Reanalysis\_1993{-}2020"}
\NormalTok{file\_nws }\OtherTok{\textless{}{-}} \StringTok{"cmems\_mod\_nws\_phy{-}bottomt\_my\_7km{-}2D\_P1M{-}m\_1677586068460.nc"}

\NormalTok{nws }\OtherTok{\textless{}{-}} \FunctionTok{read\_ncdf}\NormalTok{(}\FunctionTok{file.path}\NormalTok{(dir\_nws, file\_nws), }\AttributeTok{proxy =} \ConstantTok{TRUE}\NormalTok{) }
\end{Highlighting}
\end{Shaded}

\begin{verbatim}
## no 'var' specified, using bottomT
\end{verbatim}

\begin{verbatim}
## other available variables:
##  latitude, time, longitude
\end{verbatim}

\begin{verbatim}
## No projection information found in nc file. 
##  Coordinate variable units found to be degrees, 
##  assuming WGS84 Lat/Lon.
\end{verbatim}

\begin{Shaded}
\begin{Highlighting}[]
\CommentTok{\# Extract values of thetao at ctd points}
\CommentTok{\# change time values thetao}
\NormalTok{year\_nws }\OtherTok{\textless{}{-}} \FunctionTok{year}\NormalTok{(}\FunctionTok{attr}\NormalTok{(nws, }\StringTok{"dimensions"}\NormalTok{)}\SpecialCharTok{$}\NormalTok{time}\SpecialCharTok{$}\NormalTok{values)}
\NormalTok{month\_nws }\OtherTok{\textless{}{-}} \FunctionTok{month}\NormalTok{(}\FunctionTok{attr}\NormalTok{(nws, }\StringTok{"dimensions"}\NormalTok{)}\SpecialCharTok{$}\NormalTok{time}\SpecialCharTok{$}\NormalTok{values)}
\NormalTok{date\_nws }\OtherTok{\textless{}{-}} \FunctionTok{as.Date}\NormalTok{(}\FunctionTok{paste0}\NormalTok{(}\StringTok{"01"}\NormalTok{, }\StringTok{"{-}"}\NormalTok{, month\_nws, }\StringTok{"{-}"}\NormalTok{, year\_nws),}\AttributeTok{format =} \StringTok{"\%d{-}\%m{-}\%Y"}\NormalTok{)}

\CommentTok{\# change band dimension to time dimension}
\NormalTok{nws }\OtherTok{\textless{}{-}} \FunctionTok{st\_set\_dimensions}\NormalTok{(nws, }\DecValTok{3}\NormalTok{, }\AttributeTok{values =}\NormalTok{ date\_nws, }\AttributeTok{names =} \StringTok{"time"}\NormalTok{)}

\CommentTok{\# change time ctd\_test}
\NormalTok{ctd }\OtherTok{\textless{}{-}}\NormalTok{ ctd }\SpecialCharTok{\%\textgreater{}\%} \FunctionTok{mutate}\NormalTok{(}\AttributeTok{date\_match =} \FunctionTok{as.Date}\NormalTok{(}\FunctionTok{paste0}\NormalTok{(}\StringTok{"01"}\NormalTok{, }\StringTok{"{-}"}\NormalTok{, Month, }\StringTok{"{-}"}\NormalTok{, Year),}\AttributeTok{format =} \StringTok{"\%d{-}\%m{-}\%Y"}\NormalTok{))}

\CommentTok{\# extract }
\NormalTok{nws\_extract }\OtherTok{\textless{}{-}} \FunctionTok{st\_extract}\NormalTok{(nws, ctd, }\AttributeTok{time\_column =} \StringTok{"date\_match"}\NormalTok{)}

\CommentTok{\# joint with ctd data for comparison}
\NormalTok{nws\_join }\OtherTok{\textless{}{-}} \FunctionTok{left\_join}\NormalTok{(}\FunctionTok{as.data.frame}\NormalTok{(ctd), }\FunctionTok{as.data.frame}\NormalTok{(nws\_extract))}
\end{Highlighting}
\end{Shaded}

\begin{verbatim}
## Joining, by = c("geom", "date_match")
\end{verbatim}

\begin{Shaded}
\begin{Highlighting}[]
\NormalTok{nws\_join}\SpecialCharTok{$}\NormalTok{nws\_sbt }\OtherTok{\textless{}{-}}\NormalTok{ nws\_join}\SpecialCharTok{$}\NormalTok{bottomT}

\CommentTok{\# the rows of sbt\_join differs from ctd and sbt\_extract due to the lack of Cruise and Station information in joining}

\NormalTok{nws\_join }\OtherTok{\textless{}{-}}\NormalTok{ nws\_join }\SpecialCharTok{\%\textgreater{}\%} \FunctionTok{group\_by}\NormalTok{(ID, Cruise, Station, Year, Month, Day, sample\_date, Area\_27, date\_match, geom, depth\_m) }\SpecialCharTok{\%\textgreater{}\%} \FunctionTok{summarize}\NormalTok{(}\AttributeTok{nws\_sbt =} \FunctionTok{mean}\NormalTok{(nws\_sbt), }\AttributeTok{temp\_degC =} \FunctionTok{mean}\NormalTok{(temp\_degC))}
\end{Highlighting}
\end{Shaded}

\begin{verbatim}
## `summarise()` has grouped output by 'ID', 'Cruise', 'Station', 'Year', 'Month',
## 'Day', 'sample_date', 'Area_27', 'date_match', 'geom'. You can override using
## the `.groups` argument.
\end{verbatim}

\hypertarget{correlation-test-2}{%
\subparagraph{Correlation test}\label{correlation-test-2}}

At CTD locations

\begin{Shaded}
\begin{Highlighting}[]
\CommentTok{\# Correlation test }

\CommentTok{\# all data}
\FunctionTok{cor.test}\NormalTok{(nws\_join}\SpecialCharTok{$}\NormalTok{temp\_degC, nws\_join}\SpecialCharTok{$}\NormalTok{nws\_sbt) }
\end{Highlighting}
\end{Shaded}

\begin{verbatim}
## 
##  Pearson's product-moment correlation
## 
## data:  nws_join$temp_degC and nws_join$nws_sbt
## t = 695.46, df = 38317, p-value < 2.2e-16
## alternative hypothesis: true correlation is not equal to 0
## 95 percent confidence interval:
##  0.9618554 0.9633255
## sample estimates:
##       cor 
## 0.9625975
\end{verbatim}

\begin{Shaded}
\begin{Highlighting}[]
\CommentTok{\# by ices division}
\NormalTok{nws\_4bc }\OtherTok{\textless{}{-}}\NormalTok{ nws\_join }\SpecialCharTok{\%\textgreater{}\%} \FunctionTok{filter}\NormalTok{(Area\_27 }\SpecialCharTok{==} \StringTok{"4bc"}\NormalTok{)}
\NormalTok{nws\_7a }\OtherTok{\textless{}{-}}\NormalTok{ nws\_join }\SpecialCharTok{\%\textgreater{}\%} \FunctionTok{filter}\NormalTok{(Area\_27 }\SpecialCharTok{==} \StringTok{"7a"}\NormalTok{)}
\NormalTok{nws\_8ab }\OtherTok{\textless{}{-}}\NormalTok{ nws\_join }\SpecialCharTok{\%\textgreater{}\%} \FunctionTok{filter}\NormalTok{(Area\_27 }\SpecialCharTok{==} \StringTok{"8ab"}\NormalTok{)}

\FunctionTok{cor.test}\NormalTok{(nws\_4bc}\SpecialCharTok{$}\NormalTok{temp\_degC, nws\_4bc}\SpecialCharTok{$}\NormalTok{nws\_sbt) }
\end{Highlighting}
\end{Shaded}

\begin{verbatim}
## 
##  Pearson's product-moment correlation
## 
## data:  nws_4bc$temp_degC and nws_4bc$nws_sbt
## t = 695.59, df = 33137, p-value < 2.2e-16
## alternative hypothesis: true correlation is not equal to 0
## 95 percent confidence interval:
##  0.9667235 0.9681039
## sample estimates:
##       cor 
## 0.9674209
\end{verbatim}

\begin{Shaded}
\begin{Highlighting}[]
\FunctionTok{cor.test}\NormalTok{(nws\_7a}\SpecialCharTok{$}\NormalTok{temp\_degC, nws\_7a}\SpecialCharTok{$}\NormalTok{nws\_sbt)   }
\end{Highlighting}
\end{Shaded}

\begin{verbatim}
## 
##  Pearson's product-moment correlation
## 
## data:  nws_7a$temp_degC and nws_7a$nws_sbt
## t = 186.73, df = 1919, p-value < 2.2e-16
## alternative hypothesis: true correlation is not equal to 0
## 95 percent confidence interval:
##  0.9711282 0.9758032
## sample estimates:
##       cor 
## 0.9735674
\end{verbatim}

\begin{Shaded}
\begin{Highlighting}[]
\FunctionTok{cor.test}\NormalTok{(nws\_8ab}\SpecialCharTok{$}\NormalTok{temp\_degC, nws\_8ab}\SpecialCharTok{$}\NormalTok{nws\_sbt) }
\end{Highlighting}
\end{Shaded}

\begin{verbatim}
## 
##  Pearson's product-moment correlation
## 
## data:  nws_8ab$temp_degC and nws_8ab$nws_sbt
## t = 65.134, df = 3257, p-value < 2.2e-16
## alternative hypothesis: true correlation is not equal to 0
## 95 percent confidence interval:
##  0.7368265 0.7666693
## sample estimates:
##       cor 
## 0.7521332
\end{verbatim}

\begin{Shaded}
\begin{Highlighting}[]
\CommentTok{\# plot}
\FunctionTok{ggplot}\NormalTok{(}\AttributeTok{data =}\NormalTok{ nws\_join, }\FunctionTok{aes}\NormalTok{(}\AttributeTok{x =}\NormalTok{ temp\_degC, }\AttributeTok{y =}\NormalTok{ nws\_sbt)) }\SpecialCharTok{+} \FunctionTok{geom\_point}\NormalTok{() }\SpecialCharTok{+} \FunctionTok{geom\_smooth}\NormalTok{(}\AttributeTok{method =} \StringTok{"lm"}\NormalTok{)}
\end{Highlighting}
\end{Shaded}

\begin{verbatim}
## `geom_smooth()` using formula = 'y ~ x'
\end{verbatim}

\begin{verbatim}
## Warning: Removed 9782 rows containing non-finite values (`stat_smooth()`).
\end{verbatim}

\begin{verbatim}
## Warning: Removed 9782 rows containing missing values (`geom_point()`).
\end{verbatim}

\includegraphics{si_sbt_in-situ-ctd-vs-modeled-temperature_files/figure-latex/unnamed-chunk-14-1.pdf}

\begin{Shaded}
\begin{Highlighting}[]
\FunctionTok{ggplot}\NormalTok{(}\AttributeTok{data =}\NormalTok{ nws\_4bc, }\FunctionTok{aes}\NormalTok{(}\AttributeTok{x =}\NormalTok{ temp\_degC, }\AttributeTok{y =}\NormalTok{ nws\_sbt)) }\SpecialCharTok{+} \FunctionTok{geom\_point}\NormalTok{() }\SpecialCharTok{+} \FunctionTok{geom\_smooth}\NormalTok{(}\AttributeTok{method =} \StringTok{"lm"}\NormalTok{) }
\end{Highlighting}
\end{Shaded}

\begin{verbatim}
## `geom_smooth()` using formula = 'y ~ x'
\end{verbatim}

\begin{verbatim}
## Warning: Removed 7909 rows containing non-finite values (`stat_smooth()`).
\end{verbatim}

\begin{verbatim}
## Warning: Removed 7909 rows containing missing values (`geom_point()`).
\end{verbatim}

\includegraphics{si_sbt_in-situ-ctd-vs-modeled-temperature_files/figure-latex/unnamed-chunk-14-2.pdf}

\begin{Shaded}
\begin{Highlighting}[]
\FunctionTok{ggplot}\NormalTok{(}\AttributeTok{data =}\NormalTok{ nws\_7a, }\FunctionTok{aes}\NormalTok{(}\AttributeTok{x =}\NormalTok{ temp\_degC, }\AttributeTok{y =}\NormalTok{ nws\_sbt)) }\SpecialCharTok{+} \FunctionTok{geom\_point}\NormalTok{() }\SpecialCharTok{+} \FunctionTok{geom\_smooth}\NormalTok{(}\AttributeTok{method =} \StringTok{"lm"}\NormalTok{)}
\end{Highlighting}
\end{Shaded}

\begin{verbatim}
## `geom_smooth()` using formula = 'y ~ x'
\end{verbatim}

\begin{verbatim}
## Warning: Removed 691 rows containing non-finite values (`stat_smooth()`).
\end{verbatim}

\begin{verbatim}
## Warning: Removed 691 rows containing missing values (`geom_point()`).
\end{verbatim}

\includegraphics{si_sbt_in-situ-ctd-vs-modeled-temperature_files/figure-latex/unnamed-chunk-14-3.pdf}

\begin{Shaded}
\begin{Highlighting}[]
\FunctionTok{ggplot}\NormalTok{(}\AttributeTok{data =}\NormalTok{ nws\_8ab, }\FunctionTok{aes}\NormalTok{(}\AttributeTok{x =}\NormalTok{ temp\_degC, }\AttributeTok{y =}\NormalTok{ nws\_sbt)) }\SpecialCharTok{+} \FunctionTok{geom\_point}\NormalTok{() }\SpecialCharTok{+} \FunctionTok{geom\_smooth}\NormalTok{(}\AttributeTok{method =} \StringTok{"lm"}\NormalTok{)}
\end{Highlighting}
\end{Shaded}

\begin{verbatim}
## `geom_smooth()` using formula = 'y ~ x'
\end{verbatim}

\begin{verbatim}
## Warning: Removed 1182 rows containing non-finite values (`stat_smooth()`).
\end{verbatim}

\begin{verbatim}
## Warning: Removed 1182 rows containing missing values (`geom_point()`).
\end{verbatim}

\includegraphics{si_sbt_in-situ-ctd-vs-modeled-temperature_files/figure-latex/unnamed-chunk-14-4.pdf}

Very high correlation in the North Sea and the Irish Sea (0.96-0.97) and
high correlation in the Bay of Biscay (0.75)

Test correlation at CTD depth under 200m in the Bay of Biscay - improved
correlation 0.86

\begin{Shaded}
\begin{Highlighting}[]
\NormalTok{nws\_8ab\_st200 }\OtherTok{\textless{}{-}}\NormalTok{ nws\_8ab }\SpecialCharTok{\%\textgreater{}\%} \FunctionTok{filter}\NormalTok{(depth\_m }\SpecialCharTok{\textless{}} \DecValTok{200}\NormalTok{)}

\FunctionTok{cor.test}\NormalTok{(nws\_8ab\_st200}\SpecialCharTok{$}\NormalTok{temp\_degC, nws\_8ab\_st200}\SpecialCharTok{$}\NormalTok{nws\_sbt) }\CommentTok{\#0.86 [95CI 0.85{-}0.87]}
\end{Highlighting}
\end{Shaded}

\begin{verbatim}
## 
##  Pearson's product-moment correlation
## 
## data:  nws_8ab_st200$temp_degC and nws_8ab_st200$nws_sbt
## t = 74.22, df = 1899, p-value < 2.2e-16
## alternative hypothesis: true correlation is not equal to 0
## 95 percent confidence interval:
##  0.8503561 0.8734416
## sample estimates:
##       cor 
## 0.8623463
\end{verbatim}

\begin{Shaded}
\begin{Highlighting}[]
\FunctionTok{ggplot}\NormalTok{(}\AttributeTok{data =}\NormalTok{ nws\_8ab\_st200, }\FunctionTok{aes}\NormalTok{(}\AttributeTok{x =}\NormalTok{ temp\_degC, }\AttributeTok{y =}\NormalTok{ nws\_sbt)) }\SpecialCharTok{+} \FunctionTok{geom\_point}\NormalTok{(}\FunctionTok{aes}\NormalTok{(}\AttributeTok{color =}\NormalTok{ depth\_m)) }\SpecialCharTok{+} \FunctionTok{geom\_smooth}\NormalTok{(}\AttributeTok{method =} \StringTok{"lm"}\NormalTok{) }\SpecialCharTok{+} \FunctionTok{scale\_color\_viridis}\NormalTok{()}
\end{Highlighting}
\end{Shaded}

\begin{verbatim}
## `geom_smooth()` using formula = 'y ~ x'
\end{verbatim}

\begin{verbatim}
## Warning: Removed 715 rows containing non-finite values (`stat_smooth()`).
\end{verbatim}

\begin{verbatim}
## Warning: Removed 715 rows containing missing values (`geom_point()`).
\end{verbatim}

\includegraphics{si_sbt_in-situ-ctd-vs-modeled-temperature_files/figure-latex/unnamed-chunk-15-1.pdf}

Monthly average

\begin{Shaded}
\begin{Highlighting}[]
\CommentTok{\# correlation test aggregation by month}
\CommentTok{\# between monthly mean of ctd and monthly mean of isimip sbt}

\NormalTok{nws\_df }\OtherTok{\textless{}{-}} \FunctionTok{read\_rds}\NormalTok{(}\FunctionTok{file.path}\NormalTok{(dir\_nws, }\StringTok{"cmems\_nws\_sbt\_ices.rds"}\NormalTok{))}

\NormalTok{nws\_df }\OtherTok{\textless{}{-}}\NormalTok{ nws\_df }\SpecialCharTok{\%\textgreater{}\%}
  \FunctionTok{filter}\NormalTok{(IcesArea }\SpecialCharTok{\%in\%} \FunctionTok{c}\NormalTok{(}\StringTok{"4bc"}\NormalTok{, }\StringTok{"7a"}\NormalTok{, }\StringTok{"8ab"}\NormalTok{)) }\SpecialCharTok{\%\textgreater{}\%}
  \FunctionTok{mutate}\NormalTok{(}\AttributeTok{Month =} \FunctionTok{month}\NormalTok{(Date),}
         \AttributeTok{date\_match =} \FunctionTok{as.Date}\NormalTok{(}\FunctionTok{paste0}\NormalTok{(}\StringTok{"01"}\NormalTok{, }\StringTok{"{-}"}\NormalTok{, Month, }\StringTok{"{-}"}\NormalTok{, Year),}\AttributeTok{format =} \StringTok{"\%d{-}\%m{-}\%Y"}\NormalTok{), }
         \AttributeTok{Area\_27 =}\NormalTok{ IcesArea,}
         \AttributeTok{sbt =} \FunctionTok{as.numeric}\NormalTok{(sbt)) }

\NormalTok{nws\_join\_month }\OtherTok{\textless{}{-}}\NormalTok{ nws\_join }\SpecialCharTok{\%\textgreater{}\%} 
  \FunctionTok{group\_by}\NormalTok{(Area\_27, date\_match) }\SpecialCharTok{\%\textgreater{}\%} 
  \FunctionTok{summarize}\NormalTok{(}\AttributeTok{temp\_degC =} \FunctionTok{mean}\NormalTok{(temp\_degC, }\AttributeTok{na.ram =}\NormalTok{ T))}
\end{Highlighting}
\end{Shaded}

\begin{verbatim}
## `summarise()` has grouped output by 'Area_27'. You can override using the
## `.groups` argument.
\end{verbatim}

\begin{Shaded}
\begin{Highlighting}[]
\NormalTok{nws\_join\_month\_st200 }\OtherTok{\textless{}{-}}\NormalTok{ nws\_join }\SpecialCharTok{\%\textgreater{}\%} 
  \FunctionTok{filter}\NormalTok{(depth\_m }\SpecialCharTok{\textless{}=} \DecValTok{200}\NormalTok{) }\SpecialCharTok{\%\textgreater{}\%} \CommentTok{\#only records with depth \textless{} 200m}
  \FunctionTok{group\_by}\NormalTok{(Area\_27, date\_match) }\SpecialCharTok{\%\textgreater{}\%} 
  \FunctionTok{summarize}\NormalTok{(}\AttributeTok{temp\_degC =} \FunctionTok{mean}\NormalTok{(temp\_degC, }\AttributeTok{na.ram =}\NormalTok{ T))}
\end{Highlighting}
\end{Shaded}

\begin{verbatim}
## `summarise()` has grouped output by 'Area_27'. You can override using the
## `.groups` argument.
\end{verbatim}

\begin{Shaded}
\begin{Highlighting}[]
\NormalTok{nws\_join\_month }\OtherTok{\textless{}{-}} \FunctionTok{left\_join}\NormalTok{(nws\_join\_month, nws\_df)}
\end{Highlighting}
\end{Shaded}

\begin{verbatim}
## Joining, by = c("Area_27", "date_match")
\end{verbatim}

\begin{Shaded}
\begin{Highlighting}[]
\NormalTok{nws\_join\_month\_st200 }\OtherTok{\textless{}{-}} \FunctionTok{left\_join}\NormalTok{(nws\_join\_month\_st200, nws\_df)}
\end{Highlighting}
\end{Shaded}

\begin{verbatim}
## Joining, by = c("Area_27", "date_match")
\end{verbatim}

\begin{Shaded}
\begin{Highlighting}[]
\CommentTok{\# Correlation test}
\FunctionTok{cor.test}\NormalTok{(nws\_join\_month}\SpecialCharTok{$}\NormalTok{temp\_degC, nws\_join\_month}\SpecialCharTok{$}\NormalTok{sbt) }
\end{Highlighting}
\end{Shaded}

\begin{verbatim}
## 
##  Pearson's product-moment correlation
## 
## data:  nws_join_month$temp_degC and nws_join_month$sbt
## t = 50.7, df = 527, p-value < 2.2e-16
## alternative hypothesis: true correlation is not equal to 0
## 95 percent confidence interval:
##  0.8952436 0.9244280
## sample estimates:
##      cor 
## 0.910969
\end{verbatim}

\begin{Shaded}
\begin{Highlighting}[]
\NormalTok{nws\_month\_4bc }\OtherTok{\textless{}{-}}\NormalTok{ nws\_join\_month }\SpecialCharTok{\%\textgreater{}\%} \FunctionTok{filter}\NormalTok{(Area\_27 }\SpecialCharTok{==} \StringTok{"4bc"}\NormalTok{)}
\NormalTok{nws\_month\_7a }\OtherTok{\textless{}{-}}\NormalTok{ nws\_join\_month }\SpecialCharTok{\%\textgreater{}\%} \FunctionTok{filter}\NormalTok{(Area\_27 }\SpecialCharTok{==} \StringTok{"7a"}\NormalTok{)}
\NormalTok{nws\_month\_8ab }\OtherTok{\textless{}{-}}\NormalTok{ nws\_join\_month\_st200 }\SpecialCharTok{\%\textgreater{}\%} \FunctionTok{filter}\NormalTok{(Area\_27 }\SpecialCharTok{==} \StringTok{"8ab"}\NormalTok{)}

\FunctionTok{cor.test}\NormalTok{(nws\_month\_4bc}\SpecialCharTok{$}\NormalTok{temp\_degC, nws\_month\_4bc}\SpecialCharTok{$}\NormalTok{sbt) }
\end{Highlighting}
\end{Shaded}

\begin{verbatim}
## 
##  Pearson's product-moment correlation
## 
## data:  nws_month_4bc$temp_degC and nws_month_4bc$sbt
## t = 37.912, df = 314, p-value < 2.2e-16
## alternative hypothesis: true correlation is not equal to 0
## 95 percent confidence interval:
##  0.8839476 0.9239114
## sample estimates:
##       cor 
## 0.9059268
\end{verbatim}

\begin{Shaded}
\begin{Highlighting}[]
\FunctionTok{cor.test}\NormalTok{(nws\_month\_7a}\SpecialCharTok{$}\NormalTok{temp\_degC, nws\_month\_7a}\SpecialCharTok{$}\NormalTok{sbt)   }
\end{Highlighting}
\end{Shaded}

\begin{verbatim}
## 
##  Pearson's product-moment correlation
## 
## data:  nws_month_7a$temp_degC and nws_month_7a$sbt
## t = 29.03, df = 103, p-value < 2.2e-16
## alternative hypothesis: true correlation is not equal to 0
## 95 percent confidence interval:
##  0.918493 0.961653
## sample estimates:
##       cor 
## 0.9439774
\end{verbatim}

\begin{Shaded}
\begin{Highlighting}[]
\FunctionTok{cor.test}\NormalTok{(nws\_month\_8ab}\SpecialCharTok{$}\NormalTok{temp\_degC, nws\_month\_8ab}\SpecialCharTok{$}\NormalTok{sbt) }
\end{Highlighting}
\end{Shaded}

\begin{verbatim}
## 
##  Pearson's product-moment correlation
## 
## data:  nws_month_8ab$temp_degC and nws_month_8ab$sbt
## t = 8.0099, df = 76, p-value = 1.054e-11
## alternative hypothesis: true correlation is not equal to 0
## 95 percent confidence interval:
##  0.5345269 0.7814546
## sample estimates:
##       cor 
## 0.6765794
\end{verbatim}

\begin{Shaded}
\begin{Highlighting}[]
\CommentTok{\# plot}
\FunctionTok{ggplot}\NormalTok{(}\AttributeTok{data =}\NormalTok{ nws\_join\_month, }\FunctionTok{aes}\NormalTok{(}\AttributeTok{x =}\NormalTok{ temp\_degC, }\AttributeTok{y =}\NormalTok{ sbt)) }\SpecialCharTok{+} \FunctionTok{geom\_point}\NormalTok{() }\SpecialCharTok{+} \FunctionTok{geom\_smooth}\NormalTok{(}\AttributeTok{method =} \StringTok{"lm"}\NormalTok{)}
\end{Highlighting}
\end{Shaded}

\begin{verbatim}
## `geom_smooth()` using formula = 'y ~ x'
\end{verbatim}

\begin{verbatim}
## Warning: Removed 142 rows containing non-finite values (`stat_smooth()`).
\end{verbatim}

\begin{verbatim}
## Warning: Removed 142 rows containing missing values (`geom_point()`).
\end{verbatim}

\includegraphics{si_sbt_in-situ-ctd-vs-modeled-temperature_files/figure-latex/unnamed-chunk-16-1.pdf}

\begin{Shaded}
\begin{Highlighting}[]
\FunctionTok{ggplot}\NormalTok{(}\AttributeTok{data =}\NormalTok{ nws\_month\_4bc, }\FunctionTok{aes}\NormalTok{(}\AttributeTok{x =}\NormalTok{ temp\_degC, }\AttributeTok{y =}\NormalTok{ sbt)) }\SpecialCharTok{+} \FunctionTok{geom\_point}\NormalTok{() }\SpecialCharTok{+} \FunctionTok{geom\_smooth}\NormalTok{(}\AttributeTok{method =} \StringTok{"lm"}\NormalTok{)}
\end{Highlighting}
\end{Shaded}

\begin{verbatim}
## `geom_smooth()` using formula = 'y ~ x'
\end{verbatim}

\begin{verbatim}
## Warning: Removed 69 rows containing non-finite values (`stat_smooth()`).
\end{verbatim}

\begin{verbatim}
## Warning: Removed 69 rows containing missing values (`geom_point()`).
\end{verbatim}

\includegraphics{si_sbt_in-situ-ctd-vs-modeled-temperature_files/figure-latex/unnamed-chunk-16-2.pdf}

\begin{Shaded}
\begin{Highlighting}[]
\FunctionTok{ggplot}\NormalTok{(}\AttributeTok{data =}\NormalTok{ nws\_month\_7a, }\FunctionTok{aes}\NormalTok{(}\AttributeTok{x =}\NormalTok{ temp\_degC, }\AttributeTok{y =}\NormalTok{ sbt)) }\SpecialCharTok{+} \FunctionTok{geom\_point}\NormalTok{() }\SpecialCharTok{+} \FunctionTok{geom\_smooth}\NormalTok{(}\AttributeTok{method =} \StringTok{"lm"}\NormalTok{)}
\end{Highlighting}
\end{Shaded}

\begin{verbatim}
## `geom_smooth()` using formula = 'y ~ x'
\end{verbatim}

\begin{verbatim}
## Warning: Removed 30 rows containing non-finite values (`stat_smooth()`).
\end{verbatim}

\begin{verbatim}
## Warning: Removed 30 rows containing missing values (`geom_point()`).
\end{verbatim}

\includegraphics{si_sbt_in-situ-ctd-vs-modeled-temperature_files/figure-latex/unnamed-chunk-16-3.pdf}

\begin{Shaded}
\begin{Highlighting}[]
\FunctionTok{ggplot}\NormalTok{(}\AttributeTok{data =}\NormalTok{ nws\_month\_8ab, }\FunctionTok{aes}\NormalTok{(}\AttributeTok{x =}\NormalTok{ temp\_degC, }\AttributeTok{y =}\NormalTok{ sbt)) }\SpecialCharTok{+} \FunctionTok{geom\_point}\NormalTok{() }\SpecialCharTok{+} \FunctionTok{geom\_smooth}\NormalTok{(}\AttributeTok{method =} \StringTok{"lm"}\NormalTok{)}
\end{Highlighting}
\end{Shaded}

\begin{verbatim}
## `geom_smooth()` using formula = 'y ~ x'
\end{verbatim}

\begin{verbatim}
## Warning: Removed 28 rows containing non-finite values (`stat_smooth()`).
\end{verbatim}

\begin{verbatim}
## Warning: Removed 28 rows containing missing values (`geom_point()`).
\end{verbatim}

\includegraphics{si_sbt_in-situ-ctd-vs-modeled-temperature_files/figure-latex/unnamed-chunk-16-4.pdf}

Monthly average correlation CTD and CMEMS NWS: 0.68 {[}95CI 0.53-0.78{]}

\hypertarget{cmems-iberian-biscay-irish}{%
\paragraph{3.5. CMEMS IBERIAN BISCAY
IRISH}\label{cmems-iberian-biscay-irish}}

\hypertarget{process-data-3}{%
\subparagraph{Process data}\label{process-data-3}}

\begin{Shaded}
\begin{Highlighting}[]
\CommentTok{\# load data}
\NormalTok{dir\_ibi }\OtherTok{\textless{}{-}} \StringTok{"D:/OneDrive {-} UGent/data/Env/Sea Bottom Temperature\_CMEMS\_Atlantic{-}Iberian Biscay Irish{-} Ocean Physics Reanalysis\_1993{-}2020"}
\NormalTok{file\_ibi }\OtherTok{\textless{}{-}} \StringTok{"cmems\_mod\_ibi\_phy\_my\_0.083deg{-}3D\_P1M{-}m\_1677585944104.nc"}

\NormalTok{ibi }\OtherTok{\textless{}{-}} \FunctionTok{read\_ncdf}\NormalTok{(}\FunctionTok{file.path}\NormalTok{(dir\_ibi, file\_ibi), }\AttributeTok{proxy =} \ConstantTok{TRUE}\NormalTok{)}
\end{Highlighting}
\end{Shaded}

\begin{verbatim}
## no 'var' specified, using bottomT
\end{verbatim}

\begin{verbatim}
## other available variables:
##  latitude, time, longitude
\end{verbatim}

\begin{verbatim}
## No projection information found in nc file. 
##  Coordinate variable units found to be degrees, 
##  assuming WGS84 Lat/Lon.
\end{verbatim}

\begin{Shaded}
\begin{Highlighting}[]
\CommentTok{\# Extract values of thetao at ctd points}
\CommentTok{\# change time values thetao}
\NormalTok{year\_ibi }\OtherTok{\textless{}{-}} \FunctionTok{year}\NormalTok{(}\FunctionTok{attr}\NormalTok{(ibi, }\StringTok{"dimensions"}\NormalTok{)}\SpecialCharTok{$}\NormalTok{time}\SpecialCharTok{$}\NormalTok{values)}
\NormalTok{month\_ibi }\OtherTok{\textless{}{-}} \FunctionTok{month}\NormalTok{(}\FunctionTok{attr}\NormalTok{(ibi, }\StringTok{"dimensions"}\NormalTok{)}\SpecialCharTok{$}\NormalTok{time}\SpecialCharTok{$}\NormalTok{values)}
\NormalTok{date\_ibi }\OtherTok{\textless{}{-}} \FunctionTok{as.Date}\NormalTok{(}\FunctionTok{paste0}\NormalTok{(}\StringTok{"01"}\NormalTok{, }\StringTok{"{-}"}\NormalTok{, month\_ibi, }\StringTok{"{-}"}\NormalTok{, year\_ibi),}\AttributeTok{format =} \StringTok{"\%d{-}\%m{-}\%Y"}\NormalTok{)}

\CommentTok{\# change band dimension to time dimension}
\NormalTok{ibi }\OtherTok{\textless{}{-}} \FunctionTok{st\_set\_dimensions}\NormalTok{(ibi, }\DecValTok{3}\NormalTok{, }\AttributeTok{values =}\NormalTok{ date\_ibi, }\AttributeTok{names =} \StringTok{"time"}\NormalTok{)}

\CommentTok{\# change time ctd\_test}
\NormalTok{ctd }\OtherTok{\textless{}{-}}\NormalTok{ ctd }\SpecialCharTok{\%\textgreater{}\%} \FunctionTok{mutate}\NormalTok{(}\AttributeTok{date\_match =} \FunctionTok{as.Date}\NormalTok{(}\FunctionTok{paste0}\NormalTok{(}\StringTok{"01"}\NormalTok{, }\StringTok{"{-}"}\NormalTok{, Month, }\StringTok{"{-}"}\NormalTok{, Year),}\AttributeTok{format =} \StringTok{"\%d{-}\%m{-}\%Y"}\NormalTok{))}

\CommentTok{\# extract }
\NormalTok{ibi\_extract }\OtherTok{\textless{}{-}} \FunctionTok{st\_extract}\NormalTok{(ibi, ctd, }\AttributeTok{time\_column =} \StringTok{"date\_match"}\NormalTok{)}

\CommentTok{\# joint with ctd data for comparison}
\NormalTok{ibi\_join }\OtherTok{\textless{}{-}} \FunctionTok{left\_join}\NormalTok{(}\FunctionTok{as.data.frame}\NormalTok{(ctd), }\FunctionTok{as.data.frame}\NormalTok{(ibi\_extract))}
\end{Highlighting}
\end{Shaded}

\begin{verbatim}
## Joining, by = c("geom", "date_match")
\end{verbatim}

\begin{Shaded}
\begin{Highlighting}[]
\NormalTok{ibi\_join}\SpecialCharTok{$}\NormalTok{sbt }\OtherTok{\textless{}{-}}\NormalTok{ ibi\_join}\SpecialCharTok{$}\NormalTok{bottomT}

\CommentTok{\# the rows of sbt\_join differs from ctd and sbt\_extract due to the lack of Cruise and Station information in joining}

\NormalTok{ibi\_join }\OtherTok{\textless{}{-}}\NormalTok{ ibi\_join }\SpecialCharTok{\%\textgreater{}\%} \FunctionTok{group\_by}\NormalTok{(ID, Cruise, Station, Year, Month, Day, sample\_date, Area\_27, date\_match, geom, depth\_m) }\SpecialCharTok{\%\textgreater{}\%} \FunctionTok{summarize}\NormalTok{(}\AttributeTok{sbt =} \FunctionTok{mean}\NormalTok{(sbt), }\AttributeTok{temp\_degC =} \FunctionTok{mean}\NormalTok{(temp\_degC))}
\end{Highlighting}
\end{Shaded}

\begin{verbatim}
## `summarise()` has grouped output by 'ID', 'Cruise', 'Station', 'Year', 'Month',
## 'Day', 'sample_date', 'Area_27', 'date_match', 'geom'. You can override using
## the `.groups` argument.
\end{verbatim}

\begin{Shaded}
\begin{Highlighting}[]
\CommentTok{\# ibi data does not represent 4bc}
\NormalTok{ibi\_join }\OtherTok{\textless{}{-}}\NormalTok{ ibi\_join }\SpecialCharTok{\%\textgreater{}\%} \FunctionTok{filter}\NormalTok{(Area\_27 }\SpecialCharTok{!=} \StringTok{"4bc"}\NormalTok{)}
\end{Highlighting}
\end{Shaded}

\hypertarget{correlation-test-3}{%
\subparagraph{Correlation test}\label{correlation-test-3}}

At CTD locations

\begin{Shaded}
\begin{Highlighting}[]
\CommentTok{\# Correlation test }

\CommentTok{\# all data}
\FunctionTok{cor.test}\NormalTok{(ibi\_join}\SpecialCharTok{$}\NormalTok{temp\_degC, ibi\_join}\SpecialCharTok{$}\NormalTok{sbt) }
\end{Highlighting}
\end{Shaded}

\begin{verbatim}
## 
##  Pearson's product-moment correlation
## 
## data:  ibi_join$temp_degC and ibi_join$sbt
## t = 182.22, df = 5021, p-value < 2.2e-16
## alternative hypothesis: true correlation is not equal to 0
## 95 percent confidence interval:
##  0.9282856 0.9355556
## sample estimates:
##       cor 
## 0.9320143
\end{verbatim}

\begin{Shaded}
\begin{Highlighting}[]
\CommentTok{\# by ices division}

\NormalTok{ibi\_7a }\OtherTok{\textless{}{-}}\NormalTok{ ibi\_join }\SpecialCharTok{\%\textgreater{}\%} \FunctionTok{filter}\NormalTok{(Area\_27 }\SpecialCharTok{==} \StringTok{"7a"}\NormalTok{)}
\NormalTok{ibi\_8ab }\OtherTok{\textless{}{-}}\NormalTok{ ibi\_join }\SpecialCharTok{\%\textgreater{}\%} \FunctionTok{filter}\NormalTok{(Area\_27 }\SpecialCharTok{==} \StringTok{"8ab"}\NormalTok{)}

\FunctionTok{cor.test}\NormalTok{(ibi\_7a}\SpecialCharTok{$}\NormalTok{temp\_degC, ibi\_7a}\SpecialCharTok{$}\NormalTok{sbt)   }
\end{Highlighting}
\end{Shaded}

\begin{verbatim}
## 
##  Pearson's product-moment correlation
## 
## data:  ibi_7a$temp_degC and ibi_7a$sbt
## t = 171.03, df = 1847, p-value < 2.2e-16
## alternative hypothesis: true correlation is not equal to 0
## 95 percent confidence interval:
##  0.9670167 0.9724422
## sample estimates:
##       cor 
## 0.9698493
\end{verbatim}

\begin{Shaded}
\begin{Highlighting}[]
\FunctionTok{cor.test}\NormalTok{(ibi\_8ab}\SpecialCharTok{$}\NormalTok{temp\_degC, ibi\_8ab}\SpecialCharTok{$}\NormalTok{sbt) }
\end{Highlighting}
\end{Shaded}

\begin{verbatim}
## 
##  Pearson's product-moment correlation
## 
## data:  ibi_8ab$temp_degC and ibi_8ab$sbt
## t = 88.329, df = 3172, p-value < 2.2e-16
## alternative hypothesis: true correlation is not equal to 0
## 95 percent confidence interval:
##  0.8328191 0.8529493
## sample estimates:
##       cor 
## 0.8431795
\end{verbatim}

\begin{Shaded}
\begin{Highlighting}[]
\CommentTok{\# plot}
\FunctionTok{ggplot}\NormalTok{(}\AttributeTok{data =}\NormalTok{ ibi\_join, }\FunctionTok{aes}\NormalTok{(}\AttributeTok{x =}\NormalTok{ temp\_degC, }\AttributeTok{y =}\NormalTok{ sbt)) }\SpecialCharTok{+} \FunctionTok{geom\_point}\NormalTok{() }\SpecialCharTok{+} \FunctionTok{geom\_smooth}\NormalTok{(}\AttributeTok{method =} \StringTok{"lm"}\NormalTok{)}
\end{Highlighting}
\end{Shaded}

\begin{verbatim}
## `geom_smooth()` using formula = 'y ~ x'
\end{verbatim}

\begin{verbatim}
## Warning: Removed 2030 rows containing non-finite values (`stat_smooth()`).
\end{verbatim}

\begin{verbatim}
## Warning: Removed 2030 rows containing missing values (`geom_point()`).
\end{verbatim}

\includegraphics{si_sbt_in-situ-ctd-vs-modeled-temperature_files/figure-latex/unnamed-chunk-18-1.pdf}

\begin{Shaded}
\begin{Highlighting}[]
\FunctionTok{ggplot}\NormalTok{(}\AttributeTok{data =}\NormalTok{ ibi\_7a, }\FunctionTok{aes}\NormalTok{(}\AttributeTok{x =}\NormalTok{ temp\_degC, }\AttributeTok{y =}\NormalTok{ sbt)) }\SpecialCharTok{+} \FunctionTok{geom\_point}\NormalTok{() }\SpecialCharTok{+} \FunctionTok{geom\_smooth}\NormalTok{(}\AttributeTok{method =} \StringTok{"lm"}\NormalTok{)}
\end{Highlighting}
\end{Shaded}

\begin{verbatim}
## `geom_smooth()` using formula = 'y ~ x'
\end{verbatim}

\begin{verbatim}
## Warning: Removed 763 rows containing non-finite values (`stat_smooth()`).
\end{verbatim}

\begin{verbatim}
## Warning: Removed 763 rows containing missing values (`geom_point()`).
\end{verbatim}

\includegraphics{si_sbt_in-situ-ctd-vs-modeled-temperature_files/figure-latex/unnamed-chunk-18-2.pdf}

\begin{Shaded}
\begin{Highlighting}[]
\FunctionTok{ggplot}\NormalTok{(}\AttributeTok{data =}\NormalTok{ ibi\_8ab, }\FunctionTok{aes}\NormalTok{(}\AttributeTok{x =}\NormalTok{ temp\_degC, }\AttributeTok{y =}\NormalTok{ sbt)) }\SpecialCharTok{+} \FunctionTok{geom\_point}\NormalTok{() }\SpecialCharTok{+} \FunctionTok{geom\_smooth}\NormalTok{(}\AttributeTok{method =} \StringTok{"lm"}\NormalTok{)}
\end{Highlighting}
\end{Shaded}

\begin{verbatim}
## `geom_smooth()` using formula = 'y ~ x'
\end{verbatim}

\begin{verbatim}
## Warning: Removed 1267 rows containing non-finite values (`stat_smooth()`).
\end{verbatim}

\begin{verbatim}
## Warning: Removed 1267 rows containing missing values (`geom_point()`).
\end{verbatim}

\includegraphics{si_sbt_in-situ-ctd-vs-modeled-temperature_files/figure-latex/unnamed-chunk-18-3.pdf}

Very high correlation in the Irish Sea (0.97) and very high correlation
in the Bay of Biscay (0.84) (compared to 0.73 in CMEMS NWS data)

Test correlation at CTD depth under 200m in the Bay of Biscay - improved
correlation 0.88 (no records of very low CTD \textless{} 10 degC)

\begin{Shaded}
\begin{Highlighting}[]
\NormalTok{ibi\_8ab\_st200 }\OtherTok{\textless{}{-}}\NormalTok{ ibi\_8ab }\SpecialCharTok{\%\textgreater{}\%} \FunctionTok{filter}\NormalTok{(depth\_m }\SpecialCharTok{\textless{}} \DecValTok{200}\NormalTok{)}

\FunctionTok{cor.test}\NormalTok{(ibi\_8ab\_st200}\SpecialCharTok{$}\NormalTok{temp\_degC, ibi\_8ab\_st200}\SpecialCharTok{$}\NormalTok{sbt) }\CommentTok{\#0.50 [95CI 0.46{-}0.53]}
\end{Highlighting}
\end{Shaded}

\begin{verbatim}
## 
##  Pearson's product-moment correlation
## 
## data:  ibi_8ab_st200$temp_degC and ibi_8ab_st200$sbt
## t = 77.613, df = 1827, p-value < 2.2e-16
## alternative hypothesis: true correlation is not equal to 0
## 95 percent confidence interval:
##  0.8648340 0.8862014
## sample estimates:
##       cor 
## 0.8759466
\end{verbatim}

\begin{Shaded}
\begin{Highlighting}[]
\FunctionTok{ggplot}\NormalTok{(}\AttributeTok{data =}\NormalTok{ ibi\_8ab\_st200, }\FunctionTok{aes}\NormalTok{(}\AttributeTok{x =}\NormalTok{ temp\_degC, }\AttributeTok{y =}\NormalTok{ sbt)) }\SpecialCharTok{+} \FunctionTok{geom\_point}\NormalTok{(}\FunctionTok{aes}\NormalTok{(}\AttributeTok{color =}\NormalTok{ depth\_m)) }\SpecialCharTok{+} \FunctionTok{geom\_smooth}\NormalTok{(}\AttributeTok{method =} \StringTok{"lm"}\NormalTok{) }\SpecialCharTok{+} \FunctionTok{scale\_color\_viridis}\NormalTok{()}
\end{Highlighting}
\end{Shaded}

\begin{verbatim}
## `geom_smooth()` using formula = 'y ~ x'
\end{verbatim}

\begin{verbatim}
## Warning: Removed 787 rows containing non-finite values (`stat_smooth()`).
\end{verbatim}

\begin{verbatim}
## Warning: Removed 787 rows containing missing values (`geom_point()`).
\end{verbatim}

\includegraphics{si_sbt_in-situ-ctd-vs-modeled-temperature_files/figure-latex/unnamed-chunk-19-1.pdf}

Monthly average

\begin{Shaded}
\begin{Highlighting}[]
\CommentTok{\# correlation test aggregation by month}
\CommentTok{\# between monthly mean of ctd and monthly mean of isimip sbt}

\NormalTok{ibi\_df }\OtherTok{\textless{}{-}} \FunctionTok{read\_rds}\NormalTok{(}\FunctionTok{file.path}\NormalTok{(dir\_ibi, }\StringTok{"cmems\_ibi\_sbt\_ices.rds"}\NormalTok{))}

\NormalTok{ibi\_df }\OtherTok{\textless{}{-}}\NormalTok{ ibi\_df }\SpecialCharTok{\%\textgreater{}\%}
  \FunctionTok{filter}\NormalTok{(IcesArea }\SpecialCharTok{\%in\%} \FunctionTok{c}\NormalTok{(}\StringTok{"4bc"}\NormalTok{, }\StringTok{"7a"}\NormalTok{, }\StringTok{"8ab"}\NormalTok{)) }\SpecialCharTok{\%\textgreater{}\%}
  \FunctionTok{mutate}\NormalTok{(}\AttributeTok{Month =} \FunctionTok{month}\NormalTok{(Date),}
         \AttributeTok{date\_match =} \FunctionTok{as.Date}\NormalTok{(}\FunctionTok{paste0}\NormalTok{(}\StringTok{"01"}\NormalTok{, }\StringTok{"{-}"}\NormalTok{, Month, }\StringTok{"{-}"}\NormalTok{, Year),}\AttributeTok{format =} \StringTok{"\%d{-}\%m{-}\%Y"}\NormalTok{), }
         \AttributeTok{Area\_27 =}\NormalTok{ IcesArea,}
         \AttributeTok{sbt =} \FunctionTok{as.numeric}\NormalTok{(sbt)) }

\NormalTok{ibi\_join\_month }\OtherTok{\textless{}{-}}\NormalTok{ ibi\_join }\SpecialCharTok{\%\textgreater{}\%} 
  \FunctionTok{group\_by}\NormalTok{(Area\_27, date\_match) }\SpecialCharTok{\%\textgreater{}\%} 
  \FunctionTok{summarize}\NormalTok{(}\AttributeTok{temp\_degC =} \FunctionTok{mean}\NormalTok{(temp\_degC, }\AttributeTok{na.ram =}\NormalTok{ T))}
\end{Highlighting}
\end{Shaded}

\begin{verbatim}
## `summarise()` has grouped output by 'Area_27'. You can override using the
## `.groups` argument.
\end{verbatim}

\begin{Shaded}
\begin{Highlighting}[]
\NormalTok{ibi\_join\_month\_st200 }\OtherTok{\textless{}{-}}\NormalTok{ ibi\_join }\SpecialCharTok{\%\textgreater{}\%} 
  \FunctionTok{filter}\NormalTok{(depth\_m }\SpecialCharTok{\textless{}=} \DecValTok{200}\NormalTok{) }\SpecialCharTok{\%\textgreater{}\%} \CommentTok{\#only records with depth \textless{} 200m}
  \FunctionTok{group\_by}\NormalTok{(Area\_27, date\_match) }\SpecialCharTok{\%\textgreater{}\%} 
  \FunctionTok{summarize}\NormalTok{(}\AttributeTok{temp\_degC =} \FunctionTok{mean}\NormalTok{(temp\_degC, }\AttributeTok{na.ram =}\NormalTok{ T))}
\end{Highlighting}
\end{Shaded}

\begin{verbatim}
## `summarise()` has grouped output by 'Area_27'. You can override using the
## `.groups` argument.
\end{verbatim}

\begin{Shaded}
\begin{Highlighting}[]
\NormalTok{ibi\_join\_month }\OtherTok{\textless{}{-}} \FunctionTok{left\_join}\NormalTok{(ibi\_join\_month, ibi\_df)}
\end{Highlighting}
\end{Shaded}

\begin{verbatim}
## Joining, by = c("Area_27", "date_match")
\end{verbatim}

\begin{Shaded}
\begin{Highlighting}[]
\NormalTok{ibi\_join\_month\_st200 }\OtherTok{\textless{}{-}} \FunctionTok{left\_join}\NormalTok{(ibi\_join\_month\_st200, ibi\_df)}
\end{Highlighting}
\end{Shaded}

\begin{verbatim}
## Joining, by = c("Area_27", "date_match")
\end{verbatim}

\begin{Shaded}
\begin{Highlighting}[]
\CommentTok{\# Correlation test}
\FunctionTok{cor.test}\NormalTok{(ibi\_join\_month}\SpecialCharTok{$}\NormalTok{temp\_degC, ibi\_join\_month}\SpecialCharTok{$}\NormalTok{sbt) }\CommentTok{\#0.85 [95CI 0.83{-}0.87]}
\end{Highlighting}
\end{Shaded}

\begin{verbatim}
## 
##  Pearson's product-moment correlation
## 
## data:  ibi_join_month$temp_degC and ibi_join_month$sbt
## t = 28.769, df = 211, p-value < 2.2e-16
## alternative hypothesis: true correlation is not equal to 0
## 95 percent confidence interval:
##  0.8616315 0.9170488
## sample estimates:
##       cor 
## 0.8926653
\end{verbatim}

\begin{Shaded}
\begin{Highlighting}[]
\NormalTok{ibi\_month\_7a }\OtherTok{\textless{}{-}}\NormalTok{ ibi\_join\_month }\SpecialCharTok{\%\textgreater{}\%} \FunctionTok{filter}\NormalTok{(Area\_27 }\SpecialCharTok{==} \StringTok{"7a"}\NormalTok{)}
\NormalTok{ibi\_month\_8ab }\OtherTok{\textless{}{-}}\NormalTok{ ibi\_join\_month\_st200 }\SpecialCharTok{\%\textgreater{}\%} \FunctionTok{filter}\NormalTok{(Area\_27 }\SpecialCharTok{==} \StringTok{"8ab"}\NormalTok{)}

\FunctionTok{cor.test}\NormalTok{(ibi\_month\_7a}\SpecialCharTok{$}\NormalTok{temp\_degC, ibi\_month\_7a}\SpecialCharTok{$}\NormalTok{sbt)   }
\end{Highlighting}
\end{Shaded}

\begin{verbatim}
## 
##  Pearson's product-moment correlation
## 
## data:  ibi_month_7a$temp_degC and ibi_month_7a$sbt
## t = 30.207, df = 103, p-value < 2.2e-16
## alternative hypothesis: true correlation is not equal to 0
## 95 percent confidence interval:
##  0.9241723 0.9643808
## sample estimates:
##       cor 
## 0.9479292
\end{verbatim}

\begin{Shaded}
\begin{Highlighting}[]
\FunctionTok{cor.test}\NormalTok{(ibi\_month\_8ab}\SpecialCharTok{$}\NormalTok{temp\_degC, ibi\_month\_8ab}\SpecialCharTok{$}\NormalTok{sbt) }
\end{Highlighting}
\end{Shaded}

\begin{verbatim}
## 
##  Pearson's product-moment correlation
## 
## data:  ibi_month_8ab$temp_degC and ibi_month_8ab$sbt
## t = 7.46, df = 76, p-value = 1.182e-10
## alternative hypothesis: true correlation is not equal to 0
## 95 percent confidence interval:
##  0.4999771 0.7623951
## sample estimates:
##       cor 
## 0.6501698
\end{verbatim}

\begin{Shaded}
\begin{Highlighting}[]
\CommentTok{\# plot}
\FunctionTok{ggplot}\NormalTok{(}\AttributeTok{data =}\NormalTok{ ibi\_join\_month, }\FunctionTok{aes}\NormalTok{(}\AttributeTok{x =}\NormalTok{ temp\_degC, }\AttributeTok{y =}\NormalTok{ sbt)) }\SpecialCharTok{+} \FunctionTok{geom\_point}\NormalTok{() }\SpecialCharTok{+} \FunctionTok{geom\_smooth}\NormalTok{(}\AttributeTok{method =} \StringTok{"lm"}\NormalTok{)}
\end{Highlighting}
\end{Shaded}

\begin{verbatim}
## `geom_smooth()` using formula = 'y ~ x'
\end{verbatim}

\begin{verbatim}
## Warning: Removed 73 rows containing non-finite values (`stat_smooth()`).
\end{verbatim}

\begin{verbatim}
## Warning: Removed 73 rows containing missing values (`geom_point()`).
\end{verbatim}

\includegraphics{si_sbt_in-situ-ctd-vs-modeled-temperature_files/figure-latex/unnamed-chunk-20-1.pdf}

\begin{Shaded}
\begin{Highlighting}[]
\FunctionTok{ggplot}\NormalTok{(}\AttributeTok{data =}\NormalTok{ ibi\_month\_7a, }\FunctionTok{aes}\NormalTok{(}\AttributeTok{x =}\NormalTok{ temp\_degC, }\AttributeTok{y =}\NormalTok{ sbt)) }\SpecialCharTok{+} \FunctionTok{geom\_point}\NormalTok{() }\SpecialCharTok{+} \FunctionTok{geom\_smooth}\NormalTok{(}\AttributeTok{method =} \StringTok{"lm"}\NormalTok{)}
\end{Highlighting}
\end{Shaded}

\begin{verbatim}
## `geom_smooth()` using formula = 'y ~ x'
\end{verbatim}

\begin{verbatim}
## Warning: Removed 30 rows containing non-finite values (`stat_smooth()`).
\end{verbatim}

\begin{verbatim}
## Warning: Removed 30 rows containing missing values (`geom_point()`).
\end{verbatim}

\includegraphics{si_sbt_in-situ-ctd-vs-modeled-temperature_files/figure-latex/unnamed-chunk-20-2.pdf}

\begin{Shaded}
\begin{Highlighting}[]
\FunctionTok{ggplot}\NormalTok{(}\AttributeTok{data =}\NormalTok{ ibi\_month\_8ab, }\FunctionTok{aes}\NormalTok{(}\AttributeTok{x =}\NormalTok{ temp\_degC, }\AttributeTok{y =}\NormalTok{ sbt)) }\SpecialCharTok{+} \FunctionTok{geom\_point}\NormalTok{() }\SpecialCharTok{+} \FunctionTok{geom\_smooth}\NormalTok{(}\AttributeTok{method =} \StringTok{"lm"}\NormalTok{)}
\end{Highlighting}
\end{Shaded}

\begin{verbatim}
## `geom_smooth()` using formula = 'y ~ x'
\end{verbatim}

\begin{verbatim}
## Warning: Removed 28 rows containing non-finite values (`stat_smooth()`).
\end{verbatim}

\begin{verbatim}
## Warning: Removed 28 rows containing missing values (`geom_point()`).
\end{verbatim}

\includegraphics{si_sbt_in-situ-ctd-vs-modeled-temperature_files/figure-latex/unnamed-chunk-20-3.pdf}

Monthly average correlation CTD and CMEMS NWS: 0.65 {[}95CI 0.50-0.76{]}

\hypertarget{temperature-range}{%
\paragraph{3.6. Temperature range}\label{temperature-range}}

\begin{Shaded}
\begin{Highlighting}[]
\CommentTok{\# LOAD DATA}

\CommentTok{\# isimip thetao}
\NormalTok{dir\_thetao }\OtherTok{\textless{}{-}} \StringTok{"D:/OneDrive {-} UGent/data/Env/Sea Temperature\_ISIMIP3\_0.5deg\_1850{-}2014"}
\NormalTok{thetao\_df }\OtherTok{\textless{}{-}} \FunctionTok{read\_rds}\NormalTok{(}\FunctionTok{file.path}\NormalTok{(dir\_thetao, }\StringTok{"isimip\_sbt\_ices.rds"}\NormalTok{))}
\NormalTok{thetao\_df }\OtherTok{\textless{}{-}}\NormalTok{ thetao\_df }\SpecialCharTok{\%\textgreater{}\%} 
  \FunctionTok{mutate}\NormalTok{(}\AttributeTok{sbt =}\NormalTok{ isimip\_sbt,}
         \AttributeTok{Month =} \FunctionTok{month}\NormalTok{(Date),}
         \AttributeTok{DataSource =} \StringTok{"isimip thetao"}\NormalTok{) }\SpecialCharTok{\%\textgreater{}\%}
  \FunctionTok{select}\NormalTok{(IcesArea, sbt, Date, Month, Year, DataSource)}
\NormalTok{thetao\_df }\OtherTok{\textless{}{-}}\NormalTok{ thetao\_df }\SpecialCharTok{\%\textgreater{}\%} \FunctionTok{filter}\NormalTok{(Year }\SpecialCharTok{\textgreater{}} \DecValTok{1960}\NormalTok{, Year }\SpecialCharTok{\textless{}} \DecValTok{2022}\NormalTok{)}

\CommentTok{\# isimip tob}
\NormalTok{dir\_tob }\OtherTok{\textless{}{-}} \StringTok{"D:/OneDrive {-} UGent/data/Env/Sea Bottom Temperature\_ISIMIP3b\_MPI{-}ESM1{-}2HR\_0.5deg\_1850{-}2100"}
\NormalTok{tob\_df }\OtherTok{\textless{}{-}} \FunctionTok{read\_rds}\NormalTok{(}\FunctionTok{file.path}\NormalTok{(dir\_tob, }\StringTok{"isimip\_sbt\_ices\_hist\_ssp126.rds"}\NormalTok{))}
\NormalTok{tob\_df }\OtherTok{\textless{}{-}}\NormalTok{ tob\_df }\SpecialCharTok{\%\textgreater{}\%} \FunctionTok{mutate}\NormalTok{(}\AttributeTok{sbt =} \FunctionTok{as.numeric}\NormalTok{(isimip\_sbt),}
                            \AttributeTok{Month =} \FunctionTok{month}\NormalTok{(Date),}
                            \AttributeTok{DataSource =} \StringTok{"isimip tob"}\NormalTok{) }\SpecialCharTok{\%\textgreater{}\%}
  \FunctionTok{select}\NormalTok{(IcesArea, sbt, Date, Month, Year, DataSource)}
\NormalTok{tob\_df }\OtherTok{\textless{}{-}}\NormalTok{ tob\_df }\SpecialCharTok{\%\textgreater{}\%} \FunctionTok{filter}\NormalTok{(Year }\SpecialCharTok{\textgreater{}} \DecValTok{1960}\NormalTok{, Year }\SpecialCharTok{\textless{}} \DecValTok{2022}\NormalTok{)}


\CommentTok{\# hadisst}
\NormalTok{dir\_hadisst }\OtherTok{\textless{}{-}} \StringTok{"D:/OneDrive {-} UGent/data/Env/Sea Surface Temperature\_HadISST\_1deg\_1870{-}2021"}
\NormalTok{hadisst\_df }\OtherTok{\textless{}{-}} \FunctionTok{read\_rds}\NormalTok{(}\FunctionTok{file.path}\NormalTok{(dir\_hadisst, }\StringTok{"haidsst\_ices.rds"}\NormalTok{))}
\NormalTok{hadisst\_df }\OtherTok{\textless{}{-}}\NormalTok{ hadisst\_df }\SpecialCharTok{\%\textgreater{}\%} \FunctionTok{mutate}\NormalTok{(}\AttributeTok{sbt =}\NormalTok{ hadisst\_degC,}
                                    \AttributeTok{Month =} \FunctionTok{month}\NormalTok{(Date),}
                                    \AttributeTok{DataSource =} \StringTok{"hadisst"}\NormalTok{)  }\SpecialCharTok{\%\textgreater{}\%}
  \FunctionTok{select}\NormalTok{(IcesArea, sbt, Date, Month, Year, DataSource)}
\NormalTok{hadisst\_df }\OtherTok{\textless{}{-}}\NormalTok{ hadisst\_df }\SpecialCharTok{\%\textgreater{}\%} \FunctionTok{filter}\NormalTok{(Year }\SpecialCharTok{\textgreater{}} \DecValTok{1960}\NormalTok{, Year }\SpecialCharTok{\textless{}} \DecValTok{2022}\NormalTok{)}

\CommentTok{\# cmems nws}
\NormalTok{dir\_nws }\OtherTok{\textless{}{-}} \StringTok{"D:/OneDrive {-} UGent/data/Env/Sea Bottom Temperature\_CMEMS\_Atlantic{-} European North West Shelf{-} Ocean Physics Reanalysis\_1993{-}2020"}
\NormalTok{nws\_df }\OtherTok{\textless{}{-}} \FunctionTok{read\_rds}\NormalTok{(}\FunctionTok{file.path}\NormalTok{(dir\_nws, }\StringTok{"cmems\_nws\_sbt\_ices.rds"}\NormalTok{))}
\NormalTok{nws\_df }\OtherTok{\textless{}{-}}\NormalTok{ nws\_df }\SpecialCharTok{\%\textgreater{}\%} \FunctionTok{mutate}\NormalTok{(}\AttributeTok{Month =} \FunctionTok{month}\NormalTok{(Date))}

\CommentTok{\# cmems ibi}
\NormalTok{dir\_ibi }\OtherTok{\textless{}{-}} \StringTok{"D:/OneDrive {-} UGent/data/Env/Sea Bottom Temperature\_CMEMS\_Atlantic{-}Iberian Biscay Irish{-} Ocean Physics Reanalysis\_1993{-}2020"}
\NormalTok{ibi\_df }\OtherTok{\textless{}{-}} \FunctionTok{read\_rds}\NormalTok{(}\FunctionTok{file.path}\NormalTok{(dir\_ibi, }\StringTok{"cmems\_ibi\_sbt\_ices.rds"}\NormalTok{))}
\NormalTok{ibi\_df }\OtherTok{\textless{}{-}}\NormalTok{ ibi\_df }\SpecialCharTok{\%\textgreater{}\%} \FunctionTok{mutate}\NormalTok{(}\AttributeTok{Month =} \FunctionTok{month}\NormalTok{(Date))}

\CommentTok{\# cmems glo }
\NormalTok{dir\_glo }\OtherTok{\textless{}{-}} \StringTok{"D:/OneDrive {-} UGent/data/Env/Sea Bottom Temperature\_CMEMS\_Global Ocean Physics Reanalysis\_1993{-}2020"}
\NormalTok{glo\_df }\OtherTok{\textless{}{-}} \FunctionTok{read\_rds}\NormalTok{(}\FunctionTok{file.path}\NormalTok{(dir\_glo, }\StringTok{"cmems\_glo\_sbt\_ices.rds"}\NormalTok{))}
\NormalTok{glo\_df }\OtherTok{\textless{}{-}}\NormalTok{ glo\_df }\SpecialCharTok{\%\textgreater{}\%} \FunctionTok{mutate}\NormalTok{(}\AttributeTok{Month =} \FunctionTok{month}\NormalTok{(Date))}

\CommentTok{\# merge all modeled data}
\NormalTok{sbt }\OtherTok{\textless{}{-}} \FunctionTok{bind\_rows}\NormalTok{(thetao\_df, tob\_df, hadisst\_df, nws\_df, ibi\_df, glo\_df)}
\NormalTok{sbt }\OtherTok{\textless{}{-}}\NormalTok{ sbt }\SpecialCharTok{\%\textgreater{}\%} \FunctionTok{filter}\NormalTok{(IcesArea }\SpecialCharTok{\%in\%} \FunctionTok{c}\NormalTok{(}\StringTok{"4bc"}\NormalTok{, }\StringTok{"7a"}\NormalTok{, }\StringTok{"8ab"}\NormalTok{))}
\end{Highlighting}
\end{Shaded}

\hypertarget{temporal-trend-modeled-data}{%
\subparagraph{Temporal trend modeled
data}\label{temporal-trend-modeled-data}}

Monthly plot

\begin{Shaded}
\begin{Highlighting}[]
\CommentTok{\# Month}
\NormalTok{sbt\_month }\OtherTok{\textless{}{-}}\NormalTok{ sbt }\SpecialCharTok{\%\textgreater{}\%} \FunctionTok{group\_by}\NormalTok{(DataSource, IcesArea, Month) }\SpecialCharTok{\%\textgreater{}\%} \FunctionTok{summarize}\NormalTok{(}\AttributeTok{sbt =} \FunctionTok{mean}\NormalTok{(sbt, }\AttributeTok{na.rm =}\NormalTok{ T))}
\end{Highlighting}
\end{Shaded}

\begin{verbatim}
## `summarise()` has grouped output by 'DataSource', 'IcesArea'. You can override
## using the `.groups` argument.
\end{verbatim}

\begin{Shaded}
\begin{Highlighting}[]
\FunctionTok{ggplot}\NormalTok{(}\AttributeTok{data =}\NormalTok{ sbt\_month, }\FunctionTok{aes}\NormalTok{(}\AttributeTok{x =}\NormalTok{ Month, }\AttributeTok{y =}\NormalTok{ sbt, }\AttributeTok{color =}\NormalTok{ DataSource)) }\SpecialCharTok{+} \FunctionTok{geom\_line}\NormalTok{() }\SpecialCharTok{+} \FunctionTok{facet\_wrap}\NormalTok{(}\SpecialCharTok{\textasciitilde{}}\NormalTok{ IcesArea)}
\end{Highlighting}
\end{Shaded}

\includegraphics{si_sbt_in-situ-ctd-vs-modeled-temperature_files/figure-latex/unnamed-chunk-22-1.pdf}

\begin{Shaded}
\begin{Highlighting}[]
\FunctionTok{ggplot}\NormalTok{(}\AttributeTok{data =}\NormalTok{ sbt\_month }\SpecialCharTok{\%\textgreater{}\%} \FunctionTok{filter}\NormalTok{(IcesArea }\SpecialCharTok{==} \StringTok{"4bc"}\NormalTok{), }\FunctionTok{aes}\NormalTok{(}\AttributeTok{x =}\NormalTok{ Month, }\AttributeTok{y =}\NormalTok{ sbt, }\AttributeTok{color =}\NormalTok{ DataSource)) }\SpecialCharTok{+} \FunctionTok{geom\_line}\NormalTok{() }\SpecialCharTok{+} \FunctionTok{facet\_wrap}\NormalTok{(}\SpecialCharTok{\textasciitilde{}}\NormalTok{ IcesArea)}
\end{Highlighting}
\end{Shaded}

\includegraphics{si_sbt_in-situ-ctd-vs-modeled-temperature_files/figure-latex/unnamed-chunk-22-2.pdf}

\begin{Shaded}
\begin{Highlighting}[]
\FunctionTok{ggplot}\NormalTok{(}\AttributeTok{data =}\NormalTok{ sbt\_month }\SpecialCharTok{\%\textgreater{}\%} \FunctionTok{filter}\NormalTok{(IcesArea }\SpecialCharTok{==} \StringTok{"7a"}\NormalTok{), }\FunctionTok{aes}\NormalTok{(}\AttributeTok{x =}\NormalTok{ Month, }\AttributeTok{y =}\NormalTok{ sbt, }\AttributeTok{color =}\NormalTok{ DataSource)) }\SpecialCharTok{+} \FunctionTok{geom\_line}\NormalTok{() }\SpecialCharTok{+} \FunctionTok{facet\_wrap}\NormalTok{(}\SpecialCharTok{\textasciitilde{}}\NormalTok{ IcesArea)}
\end{Highlighting}
\end{Shaded}

\includegraphics{si_sbt_in-situ-ctd-vs-modeled-temperature_files/figure-latex/unnamed-chunk-22-3.pdf}

\begin{Shaded}
\begin{Highlighting}[]
\FunctionTok{ggplot}\NormalTok{(}\AttributeTok{data =}\NormalTok{ sbt\_month }\SpecialCharTok{\%\textgreater{}\%} \FunctionTok{filter}\NormalTok{(IcesArea }\SpecialCharTok{==} \StringTok{"8ab"}\NormalTok{), }\FunctionTok{aes}\NormalTok{(}\AttributeTok{x =}\NormalTok{ Month, }\AttributeTok{y =}\NormalTok{ sbt, }\AttributeTok{color =}\NormalTok{ DataSource)) }\SpecialCharTok{+} \FunctionTok{geom\_line}\NormalTok{() }\SpecialCharTok{+} \FunctionTok{facet\_wrap}\NormalTok{(}\SpecialCharTok{\textasciitilde{}}\NormalTok{ IcesArea) }
\end{Highlighting}
\end{Shaded}

\includegraphics{si_sbt_in-situ-ctd-vs-modeled-temperature_files/figure-latex/unnamed-chunk-22-4.pdf}

Yearly plot

\begin{Shaded}
\begin{Highlighting}[]
\CommentTok{\# Year}
\NormalTok{sbt\_year }\OtherTok{\textless{}{-}}\NormalTok{ sbt }\SpecialCharTok{\%\textgreater{}\%} \FunctionTok{group\_by}\NormalTok{(DataSource, IcesArea, Year) }\SpecialCharTok{\%\textgreater{}\%} \FunctionTok{summarize}\NormalTok{(}\AttributeTok{sbt =} \FunctionTok{mean}\NormalTok{(sbt, }\AttributeTok{na.rm =}\NormalTok{ T))}
\end{Highlighting}
\end{Shaded}

\begin{verbatim}
## `summarise()` has grouped output by 'DataSource', 'IcesArea'. You can override
## using the `.groups` argument.
\end{verbatim}

\begin{Shaded}
\begin{Highlighting}[]
\FunctionTok{ggplot}\NormalTok{(}\AttributeTok{data =}\NormalTok{ sbt\_year }\SpecialCharTok{\%\textgreater{}\%} \FunctionTok{filter}\NormalTok{(IcesArea }\SpecialCharTok{==} \StringTok{"4bc"}\NormalTok{), }\FunctionTok{aes}\NormalTok{(}\AttributeTok{x =}\NormalTok{ Year, }\AttributeTok{y =}\NormalTok{ sbt, }\AttributeTok{color =}\NormalTok{ DataSource)) }\SpecialCharTok{+} \FunctionTok{geom\_line}\NormalTok{() }\SpecialCharTok{+} \FunctionTok{facet\_wrap}\NormalTok{(}\SpecialCharTok{\textasciitilde{}}\NormalTok{ IcesArea) }\SpecialCharTok{+} \FunctionTok{geom\_smooth}\NormalTok{(}\AttributeTok{se =}\NormalTok{ F)}
\end{Highlighting}
\end{Shaded}

\begin{verbatim}
## `geom_smooth()` using method = 'loess' and formula = 'y ~ x'
\end{verbatim}

\includegraphics{si_sbt_in-situ-ctd-vs-modeled-temperature_files/figure-latex/unnamed-chunk-23-1.pdf}

\begin{Shaded}
\begin{Highlighting}[]
\FunctionTok{ggplot}\NormalTok{(}\AttributeTok{data =}\NormalTok{ sbt\_year }\SpecialCharTok{\%\textgreater{}\%} \FunctionTok{filter}\NormalTok{(IcesArea }\SpecialCharTok{==} \StringTok{"7a"}\NormalTok{), }\FunctionTok{aes}\NormalTok{(}\AttributeTok{x =}\NormalTok{ Year, }\AttributeTok{y =}\NormalTok{ sbt, }\AttributeTok{color =}\NormalTok{ DataSource)) }\SpecialCharTok{+} \FunctionTok{geom\_line}\NormalTok{() }\SpecialCharTok{+} \FunctionTok{facet\_wrap}\NormalTok{(}\SpecialCharTok{\textasciitilde{}}\NormalTok{ IcesArea) }\SpecialCharTok{+} \FunctionTok{geom\_smooth}\NormalTok{(}\AttributeTok{se =}\NormalTok{ F)}
\end{Highlighting}
\end{Shaded}

\begin{verbatim}
## `geom_smooth()` using method = 'loess' and formula = 'y ~ x'
\end{verbatim}

\includegraphics{si_sbt_in-situ-ctd-vs-modeled-temperature_files/figure-latex/unnamed-chunk-23-2.pdf}

\begin{Shaded}
\begin{Highlighting}[]
\FunctionTok{ggplot}\NormalTok{(}\AttributeTok{data =}\NormalTok{ sbt\_year }\SpecialCharTok{\%\textgreater{}\%} \FunctionTok{filter}\NormalTok{(IcesArea }\SpecialCharTok{==} \StringTok{"8ab"}\NormalTok{), }\FunctionTok{aes}\NormalTok{(}\AttributeTok{x =}\NormalTok{ Year, }\AttributeTok{y =}\NormalTok{ sbt, }\AttributeTok{color =}\NormalTok{ DataSource)) }\SpecialCharTok{+} \FunctionTok{geom\_line}\NormalTok{() }\SpecialCharTok{+} \FunctionTok{facet\_wrap}\NormalTok{(}\SpecialCharTok{\textasciitilde{}}\NormalTok{ IcesArea) }\SpecialCharTok{+} \FunctionTok{geom\_smooth}\NormalTok{(}\AttributeTok{se =}\NormalTok{ F)}
\end{Highlighting}
\end{Shaded}

\begin{verbatim}
## `geom_smooth()` using method = 'loess' and formula = 'y ~ x'
\end{verbatim}

\includegraphics{si_sbt_in-situ-ctd-vs-modeled-temperature_files/figure-latex/unnamed-chunk-23-3.pdf}

\begin{Shaded}
\begin{Highlighting}[]
\FunctionTok{ggplot}\NormalTok{(}\AttributeTok{data =}\NormalTok{ sbt\_year }\SpecialCharTok{\%\textgreater{}\%} \FunctionTok{filter}\NormalTok{(IcesArea }\SpecialCharTok{==} \StringTok{"4bc"}\NormalTok{, DataSource }\SpecialCharTok{\%in\%} \FunctionTok{c}\NormalTok{(}\StringTok{"isimip thetao"}\NormalTok{, }\StringTok{"cmems ibi"}\NormalTok{, }\StringTok{"cmems nws"}\NormalTok{)), }\FunctionTok{aes}\NormalTok{(}\AttributeTok{x =}\NormalTok{ Year, }\AttributeTok{y =}\NormalTok{ sbt, }\AttributeTok{color =}\NormalTok{ DataSource)) }\SpecialCharTok{+} \FunctionTok{geom\_line}\NormalTok{() }\SpecialCharTok{+} \FunctionTok{facet\_wrap}\NormalTok{(}\SpecialCharTok{\textasciitilde{}}\NormalTok{ IcesArea) }\SpecialCharTok{+} \FunctionTok{geom\_smooth}\NormalTok{(}\AttributeTok{se =}\NormalTok{ F)}
\end{Highlighting}
\end{Shaded}

\begin{verbatim}
## `geom_smooth()` using method = 'loess' and formula = 'y ~ x'
\end{verbatim}

\includegraphics{si_sbt_in-situ-ctd-vs-modeled-temperature_files/figure-latex/unnamed-chunk-23-4.pdf}

\begin{Shaded}
\begin{Highlighting}[]
\FunctionTok{ggplot}\NormalTok{(}\AttributeTok{data =}\NormalTok{ sbt\_year }\SpecialCharTok{\%\textgreater{}\%} \FunctionTok{filter}\NormalTok{(IcesArea }\SpecialCharTok{==} \StringTok{"7a"}\NormalTok{, DataSource }\SpecialCharTok{\%in\%} \FunctionTok{c}\NormalTok{(}\StringTok{"isimip thetao"}\NormalTok{, }\StringTok{"cmems ibi"}\NormalTok{, }\StringTok{"cmems nws"}\NormalTok{)), }\FunctionTok{aes}\NormalTok{(}\AttributeTok{x =}\NormalTok{ Year, }\AttributeTok{y =}\NormalTok{ sbt, }\AttributeTok{color =}\NormalTok{ DataSource)) }\SpecialCharTok{+} \FunctionTok{geom\_line}\NormalTok{() }\SpecialCharTok{+} \FunctionTok{facet\_wrap}\NormalTok{(}\SpecialCharTok{\textasciitilde{}}\NormalTok{ IcesArea) }\SpecialCharTok{+} \FunctionTok{geom\_smooth}\NormalTok{(}\AttributeTok{se =}\NormalTok{ F)}
\end{Highlighting}
\end{Shaded}

\begin{verbatim}
## `geom_smooth()` using method = 'loess' and formula = 'y ~ x'
\end{verbatim}

\includegraphics{si_sbt_in-situ-ctd-vs-modeled-temperature_files/figure-latex/unnamed-chunk-23-5.pdf}

\begin{Shaded}
\begin{Highlighting}[]
\FunctionTok{ggplot}\NormalTok{(}\AttributeTok{data =}\NormalTok{ sbt\_year }\SpecialCharTok{\%\textgreater{}\%} \FunctionTok{filter}\NormalTok{(IcesArea }\SpecialCharTok{==} \StringTok{"8ab"}\NormalTok{, DataSource }\SpecialCharTok{\%in\%} \FunctionTok{c}\NormalTok{(}\StringTok{"isimip thetao"}\NormalTok{, }\StringTok{"cmems ibi"}\NormalTok{, }\StringTok{"cmems nws"}\NormalTok{)), }\FunctionTok{aes}\NormalTok{(}\AttributeTok{x =}\NormalTok{ Year, }\AttributeTok{y =}\NormalTok{ sbt, }\AttributeTok{color =}\NormalTok{ DataSource)) }\SpecialCharTok{+} \FunctionTok{geom\_line}\NormalTok{() }\SpecialCharTok{+} \FunctionTok{facet\_wrap}\NormalTok{(}\SpecialCharTok{\textasciitilde{}}\NormalTok{ IcesArea) }\SpecialCharTok{+} \FunctionTok{geom\_smooth}\NormalTok{(}\AttributeTok{se =}\NormalTok{ F)}
\end{Highlighting}
\end{Shaded}

\begin{verbatim}
## `geom_smooth()` using method = 'loess' and formula = 'y ~ x'
\end{verbatim}

\includegraphics{si_sbt_in-situ-ctd-vs-modeled-temperature_files/figure-latex/unnamed-chunk-23-6.pdf}

\hypertarget{comparison-at-ctd-points}{%
\subparagraph{Comparison at CTD points}\label{comparison-at-ctd-points}}

\begin{Shaded}
\begin{Highlighting}[]
\CommentTok{\# Process data}
\NormalTok{dir\_ctd }\OtherTok{\textless{}{-}} \StringTok{"D:/OneDrive {-} UGent/data/Env/Sea Bottom Temperature\_ICES CTD\_1970{-}2020"}
\NormalTok{ctd\_df }\OtherTok{\textless{}{-}} \FunctionTok{as\_tibble}\NormalTok{(}\FunctionTok{read\_sf}\NormalTok{(}\FunctionTok{file.path}\NormalTok{(dir\_ctd, }\StringTok{"ices\_ctd\_sea{-}bottom{-}temperature.gpkg"}\NormalTok{)))}

\CommentTok{\# summarize ctd by year and month}
\NormalTok{ctd\_df }\OtherTok{\textless{}{-}}\NormalTok{ ctd\_df }\SpecialCharTok{\%\textgreater{}\%} \FunctionTok{mutate}\NormalTok{(}\AttributeTok{IcesArea =}\NormalTok{ Area\_27) }\SpecialCharTok{\%\textgreater{}\%} \FunctionTok{group\_by}\NormalTok{(IcesArea, Year, Month) }\SpecialCharTok{\%\textgreater{}\%} \FunctionTok{summarize}\NormalTok{(}\AttributeTok{ctd\_sbt =} \FunctionTok{mean}\NormalTok{(temp\_degC, }\AttributeTok{na.rm =}\NormalTok{ T))}
\end{Highlighting}
\end{Shaded}

\begin{verbatim}
## `summarise()` has grouped output by 'IcesArea', 'Year'. You can override using
## the `.groups` argument.
\end{verbatim}

\begin{Shaded}
\begin{Highlighting}[]
\NormalTok{ctd\_df }\OtherTok{\textless{}{-}}\NormalTok{ ctd\_df }\SpecialCharTok{\%\textgreater{}\%} \FunctionTok{filter}\NormalTok{(IcesArea }\SpecialCharTok{\%in\%} \FunctionTok{c}\NormalTok{(}\StringTok{"4bc"}\NormalTok{, }\StringTok{"7a"}\NormalTok{, }\StringTok{"8ab"}\NormalTok{))}

\CommentTok{\# join ctd\_df and sbt (to get modeled data at CTD points)}
\NormalTok{ctd\_df }\OtherTok{\textless{}{-}} \FunctionTok{left\_join}\NormalTok{(ctd\_df, sbt)}
\end{Highlighting}
\end{Shaded}

\begin{verbatim}
## Joining, by = c("IcesArea", "Year", "Month")
\end{verbatim}

\hypertarget{scatter-plot}{%
\subparagraph{Scatter plot}\label{scatter-plot}}

\begin{Shaded}
\begin{Highlighting}[]
\FunctionTok{ggplot}\NormalTok{(}\AttributeTok{data =}\NormalTok{ ctd\_df, }\FunctionTok{aes}\NormalTok{(}\AttributeTok{x =}\NormalTok{ ctd\_sbt, }\AttributeTok{y =}\NormalTok{ sbt)) }\SpecialCharTok{+} \FunctionTok{geom\_point}\NormalTok{() }\SpecialCharTok{+} \FunctionTok{geom\_smooth}\NormalTok{(}\AttributeTok{method =} \StringTok{"lm"}\NormalTok{) }\SpecialCharTok{+} \FunctionTok{facet\_grid}\NormalTok{(IcesArea }\SpecialCharTok{\textasciitilde{}}\NormalTok{ DataSource)}
\end{Highlighting}
\end{Shaded}

\begin{verbatim}
## `geom_smooth()` using formula = 'y ~ x'
\end{verbatim}

\includegraphics{si_sbt_in-situ-ctd-vs-modeled-temperature_files/figure-latex/unnamed-chunk-25-1.pdf}

\begin{Shaded}
\begin{Highlighting}[]
\FunctionTok{ggplot}\NormalTok{(}\AttributeTok{data =}\NormalTok{ ctd\_df }\SpecialCharTok{\%\textgreater{}\%} \FunctionTok{filter}\NormalTok{(IcesArea }\SpecialCharTok{==} \StringTok{"8ab"}\NormalTok{), }\FunctionTok{aes}\NormalTok{(}\AttributeTok{x =}\NormalTok{ ctd\_sbt, }\AttributeTok{y =}\NormalTok{ sbt)) }\SpecialCharTok{+} \FunctionTok{geom\_point}\NormalTok{() }\SpecialCharTok{+} \FunctionTok{geom\_smooth}\NormalTok{(}\AttributeTok{method =} \StringTok{"lm"}\NormalTok{) }\SpecialCharTok{+} \FunctionTok{facet\_grid}\NormalTok{(IcesArea }\SpecialCharTok{\textasciitilde{}}\NormalTok{ DataSource) }
\end{Highlighting}
\end{Shaded}

\begin{verbatim}
## `geom_smooth()` using formula = 'y ~ x'
\end{verbatim}

\includegraphics{si_sbt_in-situ-ctd-vs-modeled-temperature_files/figure-latex/unnamed-chunk-25-2.pdf}

\begin{Shaded}
\begin{Highlighting}[]
\FunctionTok{ggplot}\NormalTok{(}\AttributeTok{data =}\NormalTok{ ctd\_df }\SpecialCharTok{\%\textgreater{}\%} \FunctionTok{filter}\NormalTok{(DataSource }\SpecialCharTok{\%in\%} \FunctionTok{c}\NormalTok{(}\StringTok{"isimip thetao"}\NormalTok{, }\StringTok{"cmems ibi"}\NormalTok{, }\StringTok{"cmems nws"}\NormalTok{)), }\FunctionTok{aes}\NormalTok{(}\AttributeTok{x =}\NormalTok{ ctd\_sbt, }\AttributeTok{y =}\NormalTok{ sbt)) }\SpecialCharTok{+} \FunctionTok{geom\_point}\NormalTok{() }\SpecialCharTok{+} \FunctionTok{geom\_smooth}\NormalTok{(}\AttributeTok{method =} \StringTok{"lm"}\NormalTok{) }\SpecialCharTok{+} \FunctionTok{facet\_grid}\NormalTok{(IcesArea }\SpecialCharTok{\textasciitilde{}}\NormalTok{ DataSource)}
\end{Highlighting}
\end{Shaded}

\begin{verbatim}
## `geom_smooth()` using formula = 'y ~ x'
\end{verbatim}

\includegraphics{si_sbt_in-situ-ctd-vs-modeled-temperature_files/figure-latex/unnamed-chunk-25-3.pdf}

CTD 8ab is mainly within 10-15 degC, which fits better the range of
ISIMIP thetao and CMEMS ibi (10-13 degC)

\hypertarget{monthly-plot-1}{%
\subparagraph{Monthly plot}\label{monthly-plot-1}}

\begin{Shaded}
\begin{Highlighting}[]
\NormalTok{ctd\_month\_modeled }\OtherTok{\textless{}{-}}\NormalTok{ ctd\_df }\SpecialCharTok{\%\textgreater{}\%} 
  \FunctionTok{group\_by}\NormalTok{(DataSource, IcesArea, Month) }\SpecialCharTok{\%\textgreater{}\%} 
  \FunctionTok{summarize}\NormalTok{(}\AttributeTok{sbt =} \FunctionTok{mean}\NormalTok{(sbt, }\AttributeTok{na.rm =}\NormalTok{ T))}
\end{Highlighting}
\end{Shaded}

\begin{verbatim}
## `summarise()` has grouped output by 'DataSource', 'IcesArea'. You can override
## using the `.groups` argument.
\end{verbatim}

\begin{Shaded}
\begin{Highlighting}[]
\NormalTok{ctd\_month\_ctd }\OtherTok{\textless{}{-}}\NormalTok{ ctd\_df }\SpecialCharTok{\%\textgreater{}\%} 
  \FunctionTok{group\_by}\NormalTok{(IcesArea, Month) }\SpecialCharTok{\%\textgreater{}\%}
  \FunctionTok{summarize}\NormalTok{(}\AttributeTok{sbt =} \FunctionTok{mean}\NormalTok{(ctd\_sbt, }\AttributeTok{na.rm =}\NormalTok{ T)) }\SpecialCharTok{\%\textgreater{}\%}
  \FunctionTok{mutate}\NormalTok{(}\AttributeTok{DataSource =} \StringTok{"ices ctd"}\NormalTok{)}
\end{Highlighting}
\end{Shaded}

\begin{verbatim}
## `summarise()` has grouped output by 'IcesArea'. You can override using the
## `.groups` argument.
\end{verbatim}

\begin{Shaded}
\begin{Highlighting}[]
\NormalTok{ctd\_month }\OtherTok{\textless{}{-}} \FunctionTok{bind\_rows}\NormalTok{(ctd\_month\_modeled, ctd\_month\_ctd)}

\CommentTok{\# plot}
\FunctionTok{ggplot}\NormalTok{(}\AttributeTok{data =}\NormalTok{ ctd\_month, }\FunctionTok{aes}\NormalTok{(}\AttributeTok{x =}\NormalTok{ Month, }\AttributeTok{y =}\NormalTok{ sbt, }\AttributeTok{color =}\NormalTok{ DataSource)) }\SpecialCharTok{+} \FunctionTok{geom\_line}\NormalTok{() }\SpecialCharTok{+} \FunctionTok{facet\_wrap}\NormalTok{(}\SpecialCharTok{\textasciitilde{}}\NormalTok{ IcesArea)}
\end{Highlighting}
\end{Shaded}

\includegraphics{si_sbt_in-situ-ctd-vs-modeled-temperature_files/figure-latex/unnamed-chunk-26-1.pdf}

\begin{Shaded}
\begin{Highlighting}[]
\FunctionTok{ggplot}\NormalTok{(}\AttributeTok{data =}\NormalTok{ ctd\_month }\SpecialCharTok{\%\textgreater{}\%} \FunctionTok{filter}\NormalTok{(IcesArea }\SpecialCharTok{==} \StringTok{"4bc"}\NormalTok{), }\FunctionTok{aes}\NormalTok{(}\AttributeTok{x =}\NormalTok{ Month, }\AttributeTok{y =}\NormalTok{ sbt, }\AttributeTok{color =}\NormalTok{ DataSource)) }\SpecialCharTok{+} \FunctionTok{geom\_line}\NormalTok{() }\SpecialCharTok{+} \FunctionTok{facet\_wrap}\NormalTok{(}\SpecialCharTok{\textasciitilde{}}\NormalTok{ IcesArea)}
\end{Highlighting}
\end{Shaded}

\includegraphics{si_sbt_in-situ-ctd-vs-modeled-temperature_files/figure-latex/unnamed-chunk-26-2.pdf}

\begin{Shaded}
\begin{Highlighting}[]
\FunctionTok{ggplot}\NormalTok{(}\AttributeTok{data =}\NormalTok{ ctd\_month }\SpecialCharTok{\%\textgreater{}\%} \FunctionTok{filter}\NormalTok{(IcesArea }\SpecialCharTok{==} \StringTok{"7a"}\NormalTok{), }\FunctionTok{aes}\NormalTok{(}\AttributeTok{x =}\NormalTok{ Month, }\AttributeTok{y =}\NormalTok{ sbt, }\AttributeTok{color =}\NormalTok{ DataSource)) }\SpecialCharTok{+} \FunctionTok{geom\_line}\NormalTok{() }\SpecialCharTok{+} \FunctionTok{facet\_wrap}\NormalTok{(}\SpecialCharTok{\textasciitilde{}}\NormalTok{ IcesArea)}
\end{Highlighting}
\end{Shaded}

\includegraphics{si_sbt_in-situ-ctd-vs-modeled-temperature_files/figure-latex/unnamed-chunk-26-3.pdf}

\begin{Shaded}
\begin{Highlighting}[]
\FunctionTok{ggplot}\NormalTok{(}\AttributeTok{data =}\NormalTok{ ctd\_month }\SpecialCharTok{\%\textgreater{}\%} \FunctionTok{filter}\NormalTok{(IcesArea }\SpecialCharTok{==} \StringTok{"8ab"}\NormalTok{), }\FunctionTok{aes}\NormalTok{(}\AttributeTok{x =}\NormalTok{ Month, }\AttributeTok{y =}\NormalTok{ sbt, }\AttributeTok{color =}\NormalTok{ DataSource)) }\SpecialCharTok{+} \FunctionTok{geom\_line}\NormalTok{() }\SpecialCharTok{+} \FunctionTok{facet\_wrap}\NormalTok{(}\SpecialCharTok{\textasciitilde{}}\NormalTok{ IcesArea) }
\end{Highlighting}
\end{Shaded}

\includegraphics{si_sbt_in-situ-ctd-vs-modeled-temperature_files/figure-latex/unnamed-chunk-26-4.pdf}

\hypertarget{yearly-plot-1}{%
\subparagraph{Yearly plot}\label{yearly-plot-1}}

\begin{Shaded}
\begin{Highlighting}[]
\NormalTok{ctd\_year\_modeled }\OtherTok{\textless{}{-}}\NormalTok{ ctd\_df }\SpecialCharTok{\%\textgreater{}\%} 
  \FunctionTok{group\_by}\NormalTok{(DataSource, IcesArea, Year) }\SpecialCharTok{\%\textgreater{}\%} 
  \FunctionTok{summarize}\NormalTok{(}\AttributeTok{sbt =} \FunctionTok{mean}\NormalTok{(sbt, }\AttributeTok{na.rm =}\NormalTok{ T))}
\end{Highlighting}
\end{Shaded}

\begin{verbatim}
## `summarise()` has grouped output by 'DataSource', 'IcesArea'. You can override
## using the `.groups` argument.
\end{verbatim}

\begin{Shaded}
\begin{Highlighting}[]
\NormalTok{ctd\_year\_ctd }\OtherTok{\textless{}{-}}\NormalTok{ ctd\_df }\SpecialCharTok{\%\textgreater{}\%} 
  \FunctionTok{group\_by}\NormalTok{(IcesArea, Year) }\SpecialCharTok{\%\textgreater{}\%}
  \FunctionTok{summarize}\NormalTok{(}\AttributeTok{sbt =} \FunctionTok{mean}\NormalTok{(ctd\_sbt, }\AttributeTok{na.rm =}\NormalTok{ T)) }\SpecialCharTok{\%\textgreater{}\%}
  \FunctionTok{mutate}\NormalTok{(}\AttributeTok{DataSource =} \StringTok{"ices ctd"}\NormalTok{)}
\end{Highlighting}
\end{Shaded}

\begin{verbatim}
## `summarise()` has grouped output by 'IcesArea'. You can override using the
## `.groups` argument.
\end{verbatim}

\begin{Shaded}
\begin{Highlighting}[]
\NormalTok{ctd\_year }\OtherTok{\textless{}{-}} \FunctionTok{bind\_rows}\NormalTok{(ctd\_year\_modeled, ctd\_year\_ctd)}

\CommentTok{\# plot}
\FunctionTok{ggplot}\NormalTok{(}\AttributeTok{data =}\NormalTok{ ctd\_year }\SpecialCharTok{\%\textgreater{}\%} \FunctionTok{filter}\NormalTok{(IcesArea }\SpecialCharTok{==} \StringTok{"4bc"}\NormalTok{, }\SpecialCharTok{!}\NormalTok{DataSource }\SpecialCharTok{\%in\%} \FunctionTok{c}\NormalTok{(}\StringTok{"isimip tob"}\NormalTok{, }\StringTok{"hadisst"}\NormalTok{)), }\FunctionTok{aes}\NormalTok{(}\AttributeTok{x =}\NormalTok{ Year, }\AttributeTok{y =}\NormalTok{ sbt, }\AttributeTok{color =}\NormalTok{ DataSource)) }\SpecialCharTok{+} \FunctionTok{geom\_line}\NormalTok{() }\SpecialCharTok{+} \FunctionTok{facet\_wrap}\NormalTok{(}\SpecialCharTok{\textasciitilde{}}\NormalTok{ IcesArea)}
\end{Highlighting}
\end{Shaded}

\includegraphics{si_sbt_in-situ-ctd-vs-modeled-temperature_files/figure-latex/unnamed-chunk-27-1.pdf}

\begin{Shaded}
\begin{Highlighting}[]
\FunctionTok{ggplot}\NormalTok{(}\AttributeTok{data =}\NormalTok{ ctd\_year }\SpecialCharTok{\%\textgreater{}\%} \FunctionTok{filter}\NormalTok{(IcesArea }\SpecialCharTok{==} \StringTok{"7a"}\NormalTok{, }\SpecialCharTok{!}\NormalTok{DataSource }\SpecialCharTok{\%in\%} \FunctionTok{c}\NormalTok{(}\StringTok{"isimip tob"}\NormalTok{, }\StringTok{"hadisst"}\NormalTok{)), }\FunctionTok{aes}\NormalTok{(}\AttributeTok{x =}\NormalTok{ Year, }\AttributeTok{y =}\NormalTok{ sbt, }\AttributeTok{color =}\NormalTok{ DataSource)) }\SpecialCharTok{+} \FunctionTok{geom\_line}\NormalTok{() }\SpecialCharTok{+} \FunctionTok{facet\_wrap}\NormalTok{(}\SpecialCharTok{\textasciitilde{}}\NormalTok{ IcesArea)}
\end{Highlighting}
\end{Shaded}

\includegraphics{si_sbt_in-situ-ctd-vs-modeled-temperature_files/figure-latex/unnamed-chunk-27-2.pdf}

\begin{Shaded}
\begin{Highlighting}[]
\FunctionTok{ggplot}\NormalTok{(}\AttributeTok{data =}\NormalTok{ ctd\_year }\SpecialCharTok{\%\textgreater{}\%} \FunctionTok{filter}\NormalTok{(IcesArea }\SpecialCharTok{==} \StringTok{"8ab"}\NormalTok{, }\SpecialCharTok{!}\NormalTok{DataSource }\SpecialCharTok{\%in\%} \FunctionTok{c}\NormalTok{(}\StringTok{"isimip tob"}\NormalTok{, }\StringTok{"hadisst"}\NormalTok{)), }\FunctionTok{aes}\NormalTok{(}\AttributeTok{x =}\NormalTok{ Year, }\AttributeTok{y =}\NormalTok{ sbt, }\AttributeTok{color =}\NormalTok{ DataSource)) }\SpecialCharTok{+} \FunctionTok{geom\_line}\NormalTok{() }\SpecialCharTok{+} \FunctionTok{facet\_wrap}\NormalTok{(}\SpecialCharTok{\textasciitilde{}}\NormalTok{ IcesArea) }
\end{Highlighting}
\end{Shaded}

\includegraphics{si_sbt_in-situ-ctd-vs-modeled-temperature_files/figure-latex/unnamed-chunk-27-3.pdf}

\begin{Shaded}
\begin{Highlighting}[]
\FunctionTok{ggplot}\NormalTok{(}\AttributeTok{data =}\NormalTok{ ctd\_year }\SpecialCharTok{\%\textgreater{}\%} \FunctionTok{filter}\NormalTok{(IcesArea }\SpecialCharTok{==} \StringTok{"4bc"}\NormalTok{, DataSource }\SpecialCharTok{\%in\%} \FunctionTok{c}\NormalTok{(}\StringTok{"ices ctd"}\NormalTok{, }\StringTok{"cmems ibi"}\NormalTok{, }\StringTok{"cmems nws"}\NormalTok{, }\StringTok{"isimip thetao"}\NormalTok{)), }\FunctionTok{aes}\NormalTok{(}\AttributeTok{x =}\NormalTok{ Year, }\AttributeTok{y =}\NormalTok{ sbt, }\AttributeTok{color =}\NormalTok{ DataSource)) }\SpecialCharTok{+} \FunctionTok{geom\_line}\NormalTok{() }\SpecialCharTok{+} \FunctionTok{facet\_wrap}\NormalTok{(}\SpecialCharTok{\textasciitilde{}}\NormalTok{ IcesArea)}
\end{Highlighting}
\end{Shaded}

\includegraphics{si_sbt_in-situ-ctd-vs-modeled-temperature_files/figure-latex/unnamed-chunk-27-4.pdf}

\begin{Shaded}
\begin{Highlighting}[]
\FunctionTok{ggplot}\NormalTok{(}\AttributeTok{data =}\NormalTok{ ctd\_year }\SpecialCharTok{\%\textgreater{}\%} \FunctionTok{filter}\NormalTok{(IcesArea }\SpecialCharTok{==} \StringTok{"7a"}\NormalTok{, DataSource }\SpecialCharTok{\%in\%} \FunctionTok{c}\NormalTok{(}\StringTok{"ices ctd"}\NormalTok{, }\StringTok{"cmems ibi"}\NormalTok{, }\StringTok{"cmems nws"}\NormalTok{, }\StringTok{"isimip thetao"}\NormalTok{)), }\FunctionTok{aes}\NormalTok{(}\AttributeTok{x =}\NormalTok{ Year, }\AttributeTok{y =}\NormalTok{ sbt, }\AttributeTok{color =}\NormalTok{ DataSource)) }\SpecialCharTok{+} \FunctionTok{geom\_line}\NormalTok{() }\SpecialCharTok{+} \FunctionTok{facet\_wrap}\NormalTok{(}\SpecialCharTok{\textasciitilde{}}\NormalTok{ IcesArea)}
\end{Highlighting}
\end{Shaded}

\includegraphics{si_sbt_in-situ-ctd-vs-modeled-temperature_files/figure-latex/unnamed-chunk-27-5.pdf}

\begin{Shaded}
\begin{Highlighting}[]
\FunctionTok{ggplot}\NormalTok{(}\AttributeTok{data =}\NormalTok{ ctd\_year }\SpecialCharTok{\%\textgreater{}\%} \FunctionTok{filter}\NormalTok{(IcesArea }\SpecialCharTok{==} \StringTok{"8ab"}\NormalTok{, DataSource }\SpecialCharTok{\%in\%} \FunctionTok{c}\NormalTok{(}\StringTok{"ices ctd"}\NormalTok{, }\StringTok{"cmems ibi"}\NormalTok{, }\StringTok{"cmems nws"}\NormalTok{, }\StringTok{"isimip thetao"}\NormalTok{)), }\FunctionTok{aes}\NormalTok{(}\AttributeTok{x =}\NormalTok{ Year, }\AttributeTok{y =}\NormalTok{ sbt, }\AttributeTok{color =}\NormalTok{ DataSource)) }\SpecialCharTok{+} \FunctionTok{geom\_line}\NormalTok{() }\SpecialCharTok{+} \FunctionTok{facet\_wrap}\NormalTok{(}\SpecialCharTok{\textasciitilde{}}\NormalTok{ IcesArea) }
\end{Highlighting}
\end{Shaded}

\includegraphics{si_sbt_in-situ-ctd-vs-modeled-temperature_files/figure-latex/unnamed-chunk-27-6.pdf}

Both monthly and yearly plots show that ISIMIP thetao and CMEMS ibi have
the closest match to the CTD temperature.

\hypertarget{conclusion}{%
\paragraph{5. Conclusion}\label{conclusion}}

Among the examined datasets:

\begin{itemize}
\item
  ISIMIP thetao, ISIMIP tob
\item
  HADISST
\item
  CMEMS nws, CMEMS ibi, CMEMS glo
\end{itemize}

\textbf{ISIMIP thetao and CMEMS ibi have the closest match to the CTD in
situ temperature}

In the North Sea and the Irish Sea, both ISIMIP thetao and CMEMS ibi
show very good match in correlation test and monthly, yearly range.
ISIMIP thetao has the advantage of wider temporal coverage.

In the Bay of Biscay, CMEMS ibi shows improved correlation compared to
ISIMIP thetao at both CTD locations (0.84-0.88 vs 0.38-0.5) and at
monthly average (0.65 vs 0.52).

\textbf{Data used to represent Sea Bottom Temperature:}

\begin{itemize}
\item
  \textbf{North Sea: ISIMIP thetao}
\item
  \textbf{Irish Sea: ISIMIP thetao}
\item
  \textbf{Bay of Biscay: CMEMS ibi}
\end{itemize}

There are 2 concerns to be addressed:

\begin{enumerate}
\def\labelenumi{\arabic{enumi}.}
\tightlist
\item
  The different temporal trend of CMEMS ibi and ISIMIP thetao
\item
  The lack of temporal coverage of CMEMS ibi (dated back to 1993 only
  while otolith data can be dated back to 1980s)
\end{enumerate}

\begin{Shaded}
\begin{Highlighting}[]
\FunctionTok{ggplot}\NormalTok{(}\AttributeTok{data =}\NormalTok{ sbt\_year }\SpecialCharTok{\%\textgreater{}\%} \FunctionTok{filter}\NormalTok{(IcesArea }\SpecialCharTok{==} \StringTok{"8ab"}\NormalTok{, DataSource }\SpecialCharTok{\%in\%} \FunctionTok{c}\NormalTok{(}\StringTok{"isimip thetao"}\NormalTok{, }\StringTok{"cmems ibi"}\NormalTok{, }\StringTok{"cmems nws"}\NormalTok{)), }\FunctionTok{aes}\NormalTok{(}\AttributeTok{x =}\NormalTok{ Year, }\AttributeTok{y =}\NormalTok{ sbt, }\AttributeTok{color =}\NormalTok{ DataSource)) }\SpecialCharTok{+} \FunctionTok{geom\_line}\NormalTok{() }\SpecialCharTok{+} \FunctionTok{facet\_wrap}\NormalTok{(}\SpecialCharTok{\textasciitilde{}}\NormalTok{ IcesArea) }\SpecialCharTok{+} \FunctionTok{geom\_smooth}\NormalTok{(}\AttributeTok{se =}\NormalTok{ F)}
\end{Highlighting}
\end{Shaded}

\begin{verbatim}
## `geom_smooth()` using method = 'loess' and formula = 'y ~ x'
\end{verbatim}

\includegraphics{si_sbt_in-situ-ctd-vs-modeled-temperature_files/figure-latex/unnamed-chunk-28-1.pdf}

\end{document}
